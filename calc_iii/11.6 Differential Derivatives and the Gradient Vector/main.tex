\documentclass{article}
\usepackage[margin=1in]{geometry}
\usepackage{microtype}
\usepackage{setspace}
\usepackage{amsmath}
\usepackage{parskip}
\usepackage{amssymb}

\parskip=4ex
\date{}
\author{}

\title{11.6 Differential Derivatives and the Gradient Vector}

\begin{document}
    \maketitle
    $ D_{\vec{u}} f(x_{0},y_{0}  ), \vec{u}$ is a unit vector

    \textbf{Def}\\
    The directional derivative of $ f $ at $ (x_{0},y_{0} ) $ is the direction of a unit vector $ \hat{u}=< a, b > $ is
    \[
        \begin{gathered}
          D_{\vec{u}} f(x_{0},y_{0} ) = \lim_{h \to 0} \frac{f(x_{0}+h_{0},y_{0}+h_{0}-f(x_{0},y_{0}))}{h}  
        \end{gathered}
    \]
    if this limit exists.

    \textbf{Theorem}\\
    If $ f $ is a differentiable function of $ x ~\&~ y $, then
    \[
        \begin{gathered}
        D_{\vec{u}}f(x,y) = f_{x}(x,y)a +f_{y}(x,y)b \qquad \vec{u}=< a, b > ~\&~ \vec{u}= \text{unit vector}    
        \end{gathered}
    \]
   
    \textbf{Note}\\
    If $ \vec{u} $ makes angle $ \theta $ with the positive $ x- $axis, then $ \vec{u}=\cos{\theta},\sin{\theta} $.

    \textbf{Ex 1}\\
    Find $ D_{\vec{u}}=f(x,y)$ if $ f(x,y) =x^{3}-3xy+4y^{2} ~\&~ \hat{u}$ is the unit vector given by angle $ \theta=\frac{\pi}{6} $. What is $ D_{\vec{u}}f(1,2)$?
    \[
        \begin{gathered}
        a=\cos{\frac{\pi}{6} }=\frac{\sqrt{3}}{2} \qquad b=\sin{\frac{\pi}{6} }=\frac{1}{2} \\
        ~\\
        f_{x}(x,y)=3x^{2}-3y \qquad f_{y}(x,y)=-3x+8y\\
        ~\\
        \boxed{D_{u}f(x,y)=(3x^{2}-3y )\frac{\sqrt{3}}{2}+(-3+8y)\frac{1}{2}}\\
        ~\\
        D_{u}f(x,y)=(3(1)^{2-3(2)})\frac{\sqrt{3} }{2}+(-3(1)+8(2))\frac{1}{2}\\
        ~\\
        D_{u}f(x,y)=-\frac{3\sqrt{3}}{2} +\frac{13}{2}\\
        ~\\
        \boxed{D_{u}f(x,y)=\frac{-3\sqrt{3}+13}{2}}  
        \end{gathered}
    \]
    
    \textbf{The Gradient Vector}
    \[
        \begin{gathered}
        D_{u}=f(x,y)=f_{x}(x,y)a+f_{y}(x,y)b\\
        ~\\
        <f_{x}(x,y) ,f_{y}(x,y)> \cdot < a, b > \\
        ~\\
        D_{u}f(x,y)= \nabla f(x,y) \cdot \hat{u} \qquad \nabla f(x,y)=\text{gradient vector of }f
        \end{gathered}
    \]

    \textbf{Ex 2}\\
    Find the direcitonal derivative of the function $ f(x,y) =x^{2}y^{3}-4y  $ of the point $ (2,-1) $ in the direction of the vector $ \vec{v}=2\vec{i}+5\vec{j} $
    \[
        \begin{gathered}
        \nabla f(x,y)=< 2xy^{3}, 3x^{2}y^{2}-4>\\
        ~\\
        \nabla f(2,-1)=< 2(2)(-1)^{3} ,3(2)^{2}(-1)^{2}-4 >\\
        ~\\
        \nabla f(2,-1) = < -4, 8 >\\
        ~\\
        \hat{u}= \frac{\vec{v}}{| \vec{v} |}= \frac{< 2, 5 > }{\sqrt{29}}\\
        ~\\
        D_{u}f(2,-1)=\nabla f(x,y) \cdot \hat{u}\\
        ~\\
        D_{u}f(2,-1)= < -4, 8 > \cdot  < \frac{2}{\sqrt{29} } , \frac{5}{\sqrt{29} }  >\\
        ~\\
        \boxed{D_{\vec{u}}f(2,-1)=\frac{32}{\sqrt{29}}}     
        \end{gathered}
    \]

    \textbf{Functions of 3 Variables}\\
    For a function $ f $ of $ 3 $ variables, the gradient vector $ \nabla f $ is
    \[
        \nabla f(x,y,z)= < f_{x}(x,y,z) , f_{y}(x,y,z) , f_{z}(x,y,z) > 
    \]

    The directional derivative of $ f $ at $ (x_{0},y_{0},z_{0}) $ in the direction of a unit vector $ \hat{u} =< a, b, c > $ is
    \[
        D_{u}f(x_{0},y_{0},z_{0})= \nabla f(x_{0},y_{0},z_{)}) \cdot \hat{u} 
    \]
    
    \textbf{Ex 4A}\\
    If $ f(x,y,z) =x\sin{(yz)}$, find $ \nabla f $
    \[
        \begin{gathered}
        f_{x}(x,y,z)=\sin{y,z} \qquad f_{y}(x,y,z)=x \cdot \cos{yz} \cdot z = xz \cos{yz} \qquad f_{z}(x,y,z)= x \cdot \cos{yz} \cdot y=xy\cos{(yz)}\\
        ~\\
        \boxed{\nabla f = < \sin{(yz)}, xz\cos{(yz)}, xy\cos{yz}>} 
        \end{gathered}
    \]
    
    \textbf{Ex 4B}\\
    Find the directional derivative of $ f $ at $ (1,3,0) $ in the direction of $ \vec{v}=< 1, 2, -1 >  $
    \[
        \begin{gathered}
        \hat{u}=\frac{\vec{v}}{| \vec{v} |}= \frac{< 1, 2, -1 > }{\sqrt{6}}= < \frac{1}{\sqrt{6} } , \frac{2}{\sqrt{6} } , -\frac{1}{\sqrt{6} }  >  \\
        ~\\
        D_{u}f(x,y,z)= \nabla f(x,y,z) \cdot \hat{u}\\
        ~\\
        D_{\vec{u}}f(x,y,z) = \frac{\sin{(yz)}}{\sqrt{6} } + \frac{2xz\cos{(yz)}}{\sqrt{6} }-\frac{xy\cos{(yz)}}{\sqrt{6}}\\
        ~\\
        D_{u}f(1,3,0) = \frac{\sin{0}}{\sqrt{6} }+ 0 - \frac{3\cos{0}}{\sqrt{6}}\\
        ~\\
        \boxed{D_{u}f(1,3,0) =  -\frac{3}{\sqrt{6}}} 
        \end{gathered}
    \]

    \textbf{Maximizing the Directional Derivative}\\
    Supppouse that we have a multi-variable function $ f $ and we consider all possible directional derivatives of $ f $ at a given point. These give the rate of changes of $ f $ in all possible directions. From there, we can ask the following questions, "In which of these directions does $ f $ change the fastest" and "What is the maximum rate of change?". These questions can be answered by the folllowing theorem.

    \textbf{Theorem 15}\\
    Suppouse $ f $ is a differentiable function of two or three variables. The maximum value of the directional derivative $ D_{u}f(x)$ is $ |\nabla f(x) | $ (the magnitude of $ \nabla f(x) $ ) and it occurs when $ u $ has the same direction as the gradient vector $ \nabla f(x) $.

    \textbf{Ex 5A}\\
    If $ f(x,y) =xe^{y} $, find the rate of change of $ f $ at the point $ P(2,0) $ in the direction from $ P $ to $ Q(\frac{1}{2} ,2) $.
    \[
        \begin{gathered}
        \nabla f(x,y)=< f_{x} , f_{y} >=< e^{y} , xe^{y}  > \\
        ~\\
        \nabla (2,0)=< 1, 2 > \\
        u=\vec{PQ}=< -1.5, 2 >, \hat{u}=< -\frac{3}{5} ,\frac{4}{5}>\\
        ~\\
        D_{\hat{u}}=\nabla f(2,0) \cdot \hat{u}\\
        ~\\
        D_{\hat{u}} = < 1, 2 > \cdot < -\frac{3}{5} , \frac{4}{5}  > \\
        ~\\
      \boxed{D_{\hat{u}}=1}  
        \end{gathered}
    \]

    \textbf{Ex 5B}\\
    According to Theorem 15, $ f $ increases fastest in the direction of the gradient vector $ \nabla f(2,0)=< 1, 2 >  $. The maximum rate of change is
    \[
        | \nabla f(2,0) | = | < 1, 2 >  | = \boxed{\sqrt{5}}
    \]

    \textbf{Ex 6}\\
    Suppouse that the temperature at a point $ (x,y,z) $ in space is given by $ T(x,y,z) = \frac{80}{1+x^{2}+2y^{2}+3z^{2}} $, where $ T $ is measured in degrees Celsius and $ x,y, z $ in meters. In which direction does the temperature increase fastest at the point $ (1,1,-2) $? What is the maximum rate of increase? 
    \[
        \begin{gathered}
        \nabla T = \frac{\partial T}{\partial x}i + \frac{\partial T}{\partial y}j + \frac{\partial T}{\partial z}k\\
        ~\\
        \nabla T = - \frac{160x}{(1+x^{2}+2y^{2}+3z^{2})^{2}}i+\frac{320y}{(1+x^{2}+2y^{2}+3z^{2})^{2}}-\frac{480z}{(1+x^{2}+2y^{2}+3z^{2})^{2}}\\
        ~\\
        \nabla T=\frac{160}{(1+x^{2}+2y^{2}+3z^{2})^{2}}(-xi+2yj-3zk)\\
        \end{gathered}
    \]
    At the point $(1,1,-2) $ the gradient vector is
    \[
        \begin{gathered}
        \nabla T(1,1,-2)=\frac{160}{256}(-i-2j+6k)=\frac{5}{8}(-i-2j+6k)\\
        ~\\
        u = -i-2j+6k, \hat{u}=\frac{-1,-2,6}{\sqrt{41} }\\
        ~\\
        | \nabla T(1,1,-2) | = \frac{5}{8} | < -\frac{1}{\sqrt{41} } , -\frac{2}{\sqrt{41} } ,  \frac{6}{\sqrt{41} } > |=  
        \end{gathered}
    \]
\end{document}
