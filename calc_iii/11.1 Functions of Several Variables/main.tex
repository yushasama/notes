\documentclass{article}
\usepackage[margin=1in]{geometry}
\usepackage{microtype}
\usepackage{setspace}
\usepackage{amsmath}
\usepackage{parskip}
\usepackage{amssymb}

\parskip=4ex
\date{}
\author{}

\title{11.1 Functions of Several Variables}

\begin{document}
    \maketitle
    \textbf{Definition}\\
    A function $ f $ of two variables is a rule that assigns to each ordered pair of real numbers $ (x,y) $ in a set $ D $ a unique real number denoted by $ f(x,y) $. The set $ D $ is the domain of $ f $ and its range is the set of values $ f $ takes on, that is
    \[
      \{ f(x,y) | (x,y) \in D \} 
    \]

    At often times, $ z=f(x,y) $ is written to make explicit the value taken on by $ f $ at the general point $ (x,y) $. So $ x ~\&~ y $ are independent variables and $ z $ is the dependent variable.

    A function of two variables is just a function whose domain is a subset of $ \mathbb{R}^{2}  $ and whose range is a subset of $ \mathbb{R} $. If a function $ f $ is given by a formula and no domain is specified, then the domain of $ f $ is the set $ \{ (x,y)|x,y \in \mathbb{R} \} $.

  \textbf{Ex 1}\\
  Find the domains of the following functions and evaluate $ f(3,2) $.\\
  A) $ f(x,y)=\frac{\sqrt{x+y+1}}{x-1}$\\
  B) $ f(x,y) =x\ln(y^{2}-x) $ 

  \textbf{Ex 1A}
  \[
      \begin{gathered}
      f(x,y)=\frac{\sqrt{x+y+1}}{x-1}\\
      ~\\
      f(3,2)=\frac{\sqrt{6}}{2}\\
      ~\\ 
      D=\{ (x,y) | x+y+1\ge 0,x \neq 1\}
      \end{gathered}
  \]

  \textbf{Ex 1B}
  \[
      \begin{gathered}
      f(x,y)=x\ln(y^{2}-x)\\
      ~\\
      f(3,2)=0\\
      ~\\
      D=\{ (x,y)|y^{2}-x>0\}
      \end{gathered}
  \]

  \textbf{Ex 2}\\
  Find the domain and range of $ g(x,y)=\sqrt{9-x^{2}-y^{2}}$.
  \[
      D=\{ (x,y)|9-x^{2}-y^{2}\ge0\} \qquad R=\{ z|z=\sqrt{9-x^{2} -y^{2} },(x,y) \in D \}
  \]
  
    
\end{document}
