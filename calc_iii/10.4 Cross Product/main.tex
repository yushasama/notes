\documentclass{article}
\usepackage[margin=1in]{geometry}
\usepackage{microtype}
\usepackage{setspace}
\usepackage{amsmath}
\usepackage{parskip}
\usepackage{amssymb}
\usepackage{graphicx}

\graphicspath{{../public/}}

\parskip=4ex
\date{}
\author{}

\title{10.4 The Cross Product}

\begin{document}
  \maketitle
  \textbf{Determinant of Order 2}
  \[
    \begin{gathered}
    \begin{bmatrix}
      a & b\\
      c & d
    \end{bmatrix} = ad - bc
    \end{gathered}
  \]

  \textbf{Ex 1}
  \[
    \begin{bmatrix}
      2 &1\\
      -6 &4
    \end{bmatrix} = 2(4) -1(-6) = \boxed{14}
  \]

  \textbf{Determinant of Order 3}
  \[
    \begin{bmatrix}
      a_1 &a_2  &a_3\\
      b_1 &b_2 &b_3\\
      c_1 &c_2 &c_3
    \end{bmatrix} = 
    a_1 \begin{bmatrix}
      b_2 &b_3\\
      c_2 &c_3
    \end{bmatrix}
    - a_2 \begin{bmatrix}
      b_1 &b_3\\
      c_1 &c_3
    \end{bmatrix} +
    a_3 \begin{bmatrix}
     b_1 &b_2\\
     c_1 &c_3
    \end{bmatrix}
  \] 

  \textbf{Ex 2}
  \[
    \begin{gathered}
    \begin{bmatrix}
      1 &2 &-1\\
      3 &0 &1\\
      -5 &4 &2
    \end{bmatrix}\\
    ~\\
    1 \begin{bmatrix}
      0 &1\\
      4 &2
    \end{bmatrix} -
    2 \begin{bmatrix}
      3 &1\\
      -5 &2
    \end{bmatrix} +
    (-1) \begin{bmatrix}
      3 &0\\
      -5 &4
    \end{bmatrix}\\
    ~\\
    1(0(2)- 1(4))-2(3(2)-1(3(2)-1(-5)))+(-1)(3(4)-0(-5))\\
    ~\\
    1(-4)-2()-1(12)\to \boxed{-38}
    \end{gathered}
  \]

  \textbf{Definition}\\
  If $ \vec{ a }  = < a_1, a_2, a_3> $ and $ \vec{ b } = < b_1, b_2, 2_3>   $, then the \underline{cross product} of $ \vec{ b } ~\&~ \vec{ b }   $ is the vector.
  \[
    \vec{a} \times \vec{ b } =
    \begin{bmatrix}
      \vec{ i } & \vec{ j } & \vec{ k }\\
      a_1 &a_2 &a_3\\
      b_1 &b_2 &b_3
    \end{bmatrix}
  \]
  
  \textbf{Ex 3}\\
  If $ \vec{ a } = < 1, 3, 4>   ~ \&~ \vec{ b } = < 2, 7, -5>  $ find $ \vec{ a } \times \vec{b }$.
  \[
    \begin{gathered}
    \vec{ a } \cdot \vec{ b } =
    \begin{bmatrix}
      \vec{ i } & \vec{ j } & \vec{ k }  \\
      1 &3 &4\\
      2 &7 &-5
    \end{bmatrix}\\
    \vec{ i } 
    \begin{bmatrix}
      3 & 4\\
      7 &-5
    \end{bmatrix} -
    \vec{j} \begin{bmatrix}
      1 &4 \\
      2 &-5\\
    \end{bmatrix} +
    \vec{k} \begin{bmatrix}
      1 & 3\\
      2 & 7
    \end{bmatrix}\\
    ~\\
    \vec{i} (-15-28) - \vec{j}(-5-8) + \vec{k}(7-6)\\
    ~\\
    \boxed{-43\vec{i} + 13\vec{j} + \vec{k}} 
    \end{gathered}
  \]

  \textbf{Example 4}\\
  Show that if $ \vec{a} = <a_1, a_2, a_3> $, then $ \vec{a} \times \vec{b} =0 $
  \[
    \begin{gathered}
    \begin{bmatrix}
      \vec{i} &\vec{j} &\vec{k}\\
      a_{1} &a_{2} &a_{3}\\
      b_{1} &b_{2} &b_{3}\\ 
    \end{bmatrix}\\
    = \vec{ i } \begin{bmatrix}
      a_2 &a_3\\
      b_2 &b_3
    \end{bmatrix} - 
    \vec{ j } \begin{bmatrix}
      a_1 &a_3\\
      b_1 &b_3
    \end{bmatrix} + 
    \vec{ k } \begin{bmatrix}
      a_1 &a_2\\
      b_1 &b_1
    \end{bmatrix}\\
    ~\\
    \vec{a}(a_{2} a_{3} -a_{3} b_{2} )-\vec{j}(a_{1} a_{3} -a_{3} a_{1} )+ \vec{k}(a_{1} a_{2} a_{2} a_{1} )\\
    ~\\
    \vec{0}+\vec{0}+\vec{0}=\boxed{\vec{0}} 
    \end{gathered}
  \] 
  

  \textbf{Theorem}\\
  The vector $ \vec{ a } \cdot \vec{ b } $ is orthogonal to both vectors $ \vec{ a } ~ \&~ \vec{ b }   $. 

  \textbf{Proof}\\
  Let $ \vec{ a } = < a_1, a_2, a_3> ~ \&~ \vec{ b } = < b_1, b_2, b_3>    $. Now,
  \[
    \begin{gathered}
    \begin{bmatrix}
      \vec{ i } & \vec{ j } & \vec{ k }\\
      a_1 &a_2 &a_3\\
      b_1 &b_2 &b_3
    \end{bmatrix}
    = \vec{ i } \begin{bmatrix}
      a_2 &a_3\\
      b_2 &b_3
    \end{bmatrix} - 
    \vec{ j } \begin{bmatrix}
      a_1 &a_3\\
      b_1 &b_3
    \end{bmatrix} + 
    \vec{ k } \begin{bmatrix}
      a_1 &a_2\\
      b_1 &b_1
    \end{bmatrix}\\
    ~\\
     = a_{1}  (a_2b_3-a_3b_2) - a_{2}  (a_1b_3 - a_3b_1) + a_{3} (a_1b_2 - a_2b_{1})\\
     ~\\
     a_{1}a_{2}b_{3}-a_{1}a_{3}b_{2}-a_{2}a_{1}b_{3}-a_{3}b_{1}+a_{2}a_{3}b_{1}+ab_{1}+a_{3}a_{1}b_{2}-a_{3}a_{2}b_{1}=0\\
     ~\\
    \end{gathered}
  \]
    By similar computations, $(\vec{a}\cdot \vec{b}) \cdot \vec{b}=0$. Therefore, $ \vec{a} \cdot \vec{b} $ is orthogonal to both $ \vec{a} ~\&~ \vec{b} $.

    \textbf{Theorem}\\
    If $ \theta $ is the angle between $ \vec{a} ~\&~ \vec{b}$ (so $ 0\le \theta \le \pi $ ), then  
    \[
     | \vec{a} \cdot  \vec{b}| = \vec{a} \vec{b} \sin{\theta}
    \]
    \textbf{Corollory}\\
    Two nonzero vectors $ \vec{a} ~\&~ \vec{b}$ are parallel if and only if $ \vec{a} \cdot \vec{b} = 0 .$
    
    \textbf{Interpretation of Cross Product}\\
    Area of parallelogram $ = \text{base} \cdot  \text{height} = | \vec{a} | |\vec{b}| \sin{\theta}|=| \vec{a} \cdot  \vec{b} |$ 

    \textbf{Ex 5}\\
    Find a vector perpendicular to the plane that passes through the points $ P(1,4,6), Q(-2,5-1), ~\&~ R(1,-1,1) $. 
    \[
      \begin{aligned}
        1
      \end{aligned}
    \]
    \textbf{Ex 6}\\
    Find the area of the triangle with vectices $ P(1,4,6), Q(-2,5-1), ~\&~ R(1,-1,1) $.
    \[
      \begin{aligned}
      \text{Area}=\frac{1}{2} | \vec{PQ} \cdot  \vec{PR}|\\
      ~\\
      \frac{1}{2} \sqrt{(-40)^{2} + (-15)^{2}+(15)^{2}   }\\
      ~\\
      \frac{1}{2} \sqrt{2050}\\
      ~\\
      \frac{1}{2} \sqrt{25(82)}\\
      ~\\
      \boxed{\frac{5}{2}\sqrt{82} }
      \end{aligned}
    \]
    \textbf{Theorem}\\
    If $ \vec{a}, \vec{b} ~\&~ \vec{c}$ are vectors and $ c$ is a scalar, then
    \[
      \begin{gathered}
      1)~\vec{a} \cdot \vec{b} = -\vec{b} \cdot  \vec{a}\\
      ~\\
      2)~(c\vec{a})\\
      ~\\
      3)~ \\
      ~\\
      4)~\\
      ~\\
      5)~ \\
      ~\\
      6)~\vec{a} \cdot (\vec{b} \cdot  \vec{c}) = (\vec{a} \cdot  \vec{c})\vec{b} - (\vec{a} \cdot  \vec{b})\vec{c}
      \end{gathered}
    \]

    \textbf{Triple Products}
    \[
      \begin{gathered}
      \vec{a} \cdot  (\vec{b} \cdot \vec{c}) = 
      \begin{bmatrix}
        a_{1} &a_{2} &a_{3}\\
        b_{2} &b_{2} &b_{3}\\
        c_{1} &c_{2} &c_{2} 
      \end{bmatrix}\\
      ~\\
      V = (\text{Area of Base})(\text{Height})\\
      ~\\
      | \vec{b} \cdot \vec{c} | \cdot  | \vec{a} | \cdot  | \cos{\theta} |\\
      ~\\
      | \vec{a} \cdot  (\vec{b} \cdot  \vec{c}) |
      \end{gathered}
    \]
    \textbf{Note}\\
    If $ \vec{a} \cdot (\vec{b} \cdot  \vec{c}) = 0$, then the vectors are coplanar (lie on the same plane).

    \textbf{Ex 7}\\
    If $ \vec{a} = < 1, 4, -7>, \vec{b}= < 2, -1, 4> ~\&~ \vec{c}= < 0, -9, 18>  $, then
    \[
      \begin{gathered}
      \vec{a} \cdot  (\vec{b} \cdot  \vec{c}) = 
      \begin{bmatrix}
        1 &4 &-7\\
        2 &-1 &4\\
        0 &-9 &18
      \end{bmatrix}\\
      ~\\
      1(-18-(-36))-4(36-0)-7(-18-0)\\
      ~\\
      18-144+126\\
      ~\\
      0
      \end{gathered}
    \]
    Thus, $ \vec{a}, \vec{b}, ~\&~\vec{c} $ are coplanar.

    \textbf{Torque}\\
    Consider a force $ \vec{F} $ acting on a rigid body at a point given by a position vector $\vec{r}$. The torque $ \tau $ (relative to the origin) is
    \[
      \vec{\tau}=\vec{r} \cdot  \vec{F}
    \]
    and measures the tendency of the body to rotate about the origin. THe magnitude of the torque vector is
    \[
    | \vec{\tau}| =| \vec{r} \cdot  \vec{F} | = |\vec{r}| \cdot  | \vec{F} | \cdot \sin{\theta}
    \]
    where $ \theta $ is the angle between $ \vec{r} ~\&~\vec{F} $.

    \textbf{Ex 8}\\
    A bolt is tightened by applying a $ 40N $ force to a $ 0.25m $ wrench. Find the magnitude of the torque about the center of the bolt. 

    \[
      \begin{gathered}
      | \tau | = 0.25(40) \sin{75^o}\\
      ~\\
      \boxed{9.66N\cdot m}
      \end{gathered}
    \]
    
    
\end{document} 
