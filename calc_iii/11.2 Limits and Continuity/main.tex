\documentclass{article}
\usepackage[margin=1in]{geometry}
\usepackage{microtype}
\usepackage{setspace}
\usepackage{amsmath}
\usepackage{parskip}
\usepackage{amssymb}
\usepackage{graphicx}

\graphicspath{{../public/}}

\parskip=4ex
\date{}
\author{}

\title{11.2 Limits and Continuity}

\begin{document}
  \maketitle
  \textbf{Introduction}\\
  The limit of multi-variable functions is similar to the limit of a function of a single variable.
  \[
    \lim_{(x,y)\to(a,b)}f(x,y)=L 
  \]

  This notation indicates that the values of $ f(x,y) $ approach the number $ L $ as the point $ (x,y) $ approaches $ (a,b) $ along any path that is within the domain of $ f $.
  
 \textbf{Ex 1}\\
 Show that $ \lim_{(x,y)\to(0,0)}\frac{x^{2}-y^{2}}{x^{2}+y^{2}}$ does not exist.
 \[
   \begin{gathered}
   \lim_{(x,y) \to (0,0)} \frac{x^{2}-y^{2}}{x^{2}+y^{2}}=\frac{x^{2}}{x^{2}}=1 \qquad \lim_{(x,y) \to (0,0)} \frac{x^{2}-y^{2}}{x^{2}+y^{2}}=\frac{-y^{2}}{y^{2}}=-1\\
   ~\\
   y=0,\text{ along the x axis} \qquad x=0, \text{along the y axis}\\
   ~\\
   \lim_{(x,y)\to(0,0)} \frac{x^{2}-y^{2}}{x^{2}+y^{2}} \text{ (along $ x $ axis )} \neq \lim_{(x,y)\to(0,0)} \frac{x^{2}-y^{2}}{x^{2}+y^{2}} \text{ (along $ y $ axis)}   
   \end{gathered}
 \]

 Since $ f $ has two different limits along two different lines, the given limit does not exist.

 \textbf{Ex 2}\\
 If $ f(x,y)  = \frac{xy}{x^{2}+y^{2}  }$, does $ \lim_{(x,y)\to(0,0)} f(x,y) $ exist?
 \[
   \begin{gathered}
   \lim_{(x,y)\to (0,0)} \frac{xy}{x^{2}+y^{2}}=\frac{0}{x^{2}} = 0 \qquad \qquad \lim_{(x,y)\to (0,0)} \frac{xy}{x^{2}+y^{2}}=\frac{0}{y^{2} }=0\\
   ~\\
   y=0,\text{ along the x axis} \qquad x=0, \text{along the y axis}\\
   ~\\
   \lim_{(x,y)\to (0,0)} \frac{xy}{x^{2}+y^{2}} \text{ (along $ x $ axis)} =  \lim_{(x,y)\to (0,0)} \frac{xy}{x^{2}+y^{2}} \text{ (along $ y $ axis)} 
   \end{gathered}
 \]

 Although $ f $ have identical limits along two different lines, we have still yet to show that the given limit is $ 0 $. Let's approach $ (0,0) $ along another line such as $ y=x, x\neq0 $.
 \[
   \begin{gathered}
   \lim_{(x,y)\to x,x} \frac{xy}{x^{2}+y^{2}} =\frac{x^{2}}{2x^{2}}=\frac{1}{2}\\
   \end{gathered}
 \]

 Since we have obtained different limits along different paths, the given limit does not exist.

 \textbf{Ex 3}\\
 If $ f(x,y)=\frac{xy^{2}}{x^{2}+y^{4}} $, does $ \lim_{(x,y) \to (0,0)}f(x,y) $ exist?
 Let's save time by letting $ (x,y)\to(0,0) $ along any nonverticle line through the origin. Then $ y=mx ~\&~ m$  is the slope.
 \[
   \begin{gathered}
     f(x,y)=f(x,mx)=\frac{x(mx)^{2}}{x^{2}+(mx)^{4}}=\frac{m^{2}x^{3}}{x^{2}+m^{4}x^{4}}=\frac{m^{2}x^{3}}{x^{2}(1+m^{4}x^{2})}=\frac{m^{2}x}{1+m^{4}x^{2}}\\
     ~\\
     f(x,y)\to 0 \text{($ x,y $ along $ y=mx $ )}
   \end{gathered}
 \]

 $ f $ has the same limiting value along every nonverticle line through the origin. However that does not show that the given limit is $ 0 $. Let us find the limit along the parabola $ x=y^{2}$
 \[
   \begin{gathered}
     f(x,y)=f(y^{2},y)=\frac{y^{2} \cdot y^{2}  }{(y^{2})^{2} +y^{4} }=\frac{y^{4}}{2y^{4}}=\frac{1}{2}\\
     ~\\
     f(x,y)\to \frac{1}{2} \text{ as $ (x,y)\to (0,0) $ along $ x=y^{2} $ } 
   \end{gathered}
 \]
 Because different paths lead to different limiting values, the given limit does not exist.

 The calculation of limits for functions of two variables can be simplified through the use of properties of limits.

 \textbf{Limit Laws}
 \[
   \begin{gathered}
   \lim_{(x,y) \to (a,b)} x=a \qquad \lim_{(x,y) \to (a,b)} y=b \qquad \lim_{(x,y) \to (a,b)} c=c\\
   ~\\
   c=\mathbb{R}
   \end{gathered}
 \]

 \textbf{Ex 4}\\
 Find the limit, if it exists
 \[
   \begin{gathered}
   \lim_{(x,y) \to (0,0)} = \frac{xy}{\sqrt{x^{2}+y^{2}}}\\
   ~\\
   \lim_{(x,y) \to (0,0)}f(x,y) \qquad y=0,\text{ along the x axis}\\
   ~\\
   \lim_{(x,y) \to (0,0)}\frac{0}{x^{2}+0 } \to 0\\
   ~\\
   \lim_{(x,y) \to (0,0)}f(x,y) \qquad y=0,\text{ along the y axis}\\
   ~\\
   \lim_{(x,y) \to (0,0)}\frac{0}{x^{2}+0 } \to 0
   \end{gathered}
 \]

 Now because the limits are the same when approaching both the $ x ~\&~ y$ axises, we will examine another approach path. Approach $ (0,0) $ along the curve $ y=x^{2}$.
 \[
   \begin{gathered}
    \lim_{(x,y) \to (0,0)}f(x,y)=\frac{x^{2}y^{2}}{\sqrt{x^{2}+x^{4}}}=\frac{x^{3}}{x^{2}(1+x^{2})}=\frac{x^{2}}{\sqrt{1+x^{2}}}\to0
  \end{gathered}
 \]

 Now it seems possible that the limit does indeed exist and equal $ 0 $. This can be proven by applying the Squeeze Theorem, $ g(x)\le f(x) \le h(x) $ for every $ x $ in an interval. If $ \lim_{x \to a} g(x) = \lim_{x \to a} h(x)=L$, then $ \lim_{x \to a} f(x)=L $.
 \[
   \begin{gathered}
     0 \le | \frac{xy}{\sqrt{x^{2}+y^{2}}} |, \text{ this holds true due to properties of  absolute value}\\
     ~\\
     | \frac{xy}{\sqrt{x^{2}+y^{2}}} |\to \frac{| x || y |}{\sqrt{x^{2}+y^{2}}}\\
     ~\\  
     \sqrt{y^{2}}=| y | \le \sqrt{x^{2}+y^{2} }, \text{ this holds true since $ x^{2}  $ is always a positive value }\\
     ~\\
     0 \le \frac{| x || y |}{\sqrt{x^{2}+y^{2}}} \le \frac{|x|\sqrt{x^{2}+y^{2}}}{\sqrt{x^{2}+y^{2}}}\\
     ~\\
     0 \le \frac{| x || y |}{\sqrt{x^{2} +y^{2} } } \le | x |\\
     ~\\
     \lim_{(x,y) \to (0,0)} 0  = \lim_{(x,y) \to (0,0)} | x |\\
     ~\\
     0 = 0
   \end{gathered}
 \]

 Since the limits of $ g(x) ~\&~ h(x) $ or respectively $ 0 ~\&~ | x | $ are the same, then by the Squeeze Theorem, the limit of $ \frac{| x || y |}{\sqrt{x^{2}+y^{2}  } }  $ must also equal 0. So then
 \[
  \boxed{\lim_{(x,y) \to (0,0)}\frac{xy}{\sqrt{x^{2} +y^{2} } } = 0} 
 \]

 \textbf{Continuity}\\
 A function $ f $ of two variables is called continuous at $ (a,b) $ if
 \[
   \lim_{(x,y) \to (a,b)}f(x,y)=f(a,b) 
 \]
 $ f $ is continuous on $ D $ if $ f $ is continuous at every point $ (a,b) $ in $ D $.

 A polynomial function of two variables or polynomial is a sum of terms of the form $ cx^{m}y^{n} $ where $ c $ is a constant and $ m ~\&~ n $ are nonnegative integers. A rational function is a ratio of polynomials.

 \textbf{Ex 5}\\
 Evaluate $ \lim_{(x,y) \to (1,2)} x^{2}y^{3}-x^{3}y^{2}+3x+2y $.
 \[
   \boxed{\lim_{(x,y) \to (1,2)} x^{2}y^{3}-x^{3}y^{2}+3x+2y=11}
 \]

 \textbf{Ex 6}\\
 Where is the function $ f(x,y)=\frac{x^{2}-y^{2}}{x^{2}+y^{2}} $?
 \[
   D=\{ (x,y)|(x,y)\neq (0,0) \}
 \]

 The function $ f $ is discontinuous at $ (0,0) $ because it is not defined there. Due to $ f $ being a rational function, it is continuous on its domain as shown in the set above.

 \textbf{Ex 9}\\
 Where is the function $ h(x,y) = \arctan{(\frac{y}{x})} $ continuous?
 \[
   \begin{gathered}
   f(x,y)=\frac{y}{x},~D=\{ (x,y)| x\neq 0 \}\\
   ~\\
   g(t)=\arctan t,~D=\{ t|t \in \mathbb{R} \}\\
   ~\\
   g(f(x,y))=\arctan{(\frac{y}{x})}=h(x,y)\\
   ~\\
   \boxed{D=\{ (x,y) | x \neq 0 \}} 
   \end{gathered}
 \]
 The function $ h(x,y)=\arctan{(\frac{y}{x} )} $ is discontinuous where $ x=0 $. 
 
\end{document}
