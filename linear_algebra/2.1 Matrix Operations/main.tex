\documentclass{article}
\usepackage[margin=1in]{geometry}
\usepackage{microtype}
\usepackage{setspace}
\usepackage{amsmath}
\usepackage{parskip}
\usepackage{amssymb}
\usepackage{graphicx}

\graphicspath{{../public/}}

\parskip=4ex
\date{}
\author{}

\title{2.1 Matrix Operations}

\begin{document}
  \maketitle
  Imagine two matrixes $ A ~\&~B $, in order for $ A+B $ to be possible, they must both be $ m \times n $ matrixes, meaning they have the same dimensions.

  \textbf{Matrix Addition}
  \[
    \begin{gathered}
    A,~m \times n \qquad B m \times n\\
    A+B,~m \times n
    \end{gathered}
  \]

  \textbf{Matrix Multiplication}
  \[
    \begin{gathered}
    A,~m \times n \qquad B, n \times p\\
    A \cdot B,~m \times p
    \end{gathered}
  \]

  \textbf{Ex 3 p101}\\
  Compute $ AB $ where \[
    \begin{gathered}
    A = \begin{bmatrix}
      2 &3\\
      1 &-5
    \end{bmatrix} \qquad
    B = \begin{bmatrix}
      4 &3 &6\\
      1 &-2 &3
    \end{bmatrix}\\
    ~\\
    A,~2\times2 \qquad B,~2 \times 3\\
    ~\\
    \begin{bmatrix}
      A_{r1} \cdot B_{c1} &A_{r1} \cdot B_{c1} &A_{r1} \cdot B_{c2}\\
      A_{r2} \cdot B_{c1} &A_{r2} \cdot B_{c2} &A_{r2} \cdot B_{c3}    
    \end{bmatrix} \to
    \begin{bmatrix}
      4(2) +3(1) &2(3) + (3)(-2) &2(6) + (3)(3)\\
      1(1) + (-5)(1) &1(3) + (-5)(-2) &1(6) + (-5)(3)
    \end{bmatrix}\\
    ~\\
    \begin{bmatrix}
      11 &0 &21\\
      01 &13 &-9
    \end{bmatrix}
    \end{gathered}
  \]

    \textbf{Ex 6 p103}\\
    Find the entries in the second row of $ AB $, where
    \[
      \begin{gathered}
      A= \begin{bmatrix}
        2 &-5 &0\\
        -1 &3 &-4\\
        6 &-8 &-7\\
        -3 &0 &9
      \end{bmatrix}\qquad
      B=\begin{bmatrix}
        4 &-6\\
        7 &1\\
        3 &2
      \end{bmatrix}\\
      ~\\
      A,~4 \times 3 \qquad B,~3 \times 2\\
      ~\\
      \begin{bmatrix}
        A_{r1} \cdot B_{c1} &A_{r1} \cdot B_{c2}\\
        A_{r2} \cdot B_{c1} &A_{r2} \cdot B_{c2}\\
        A_{r3} \cdot B_{c1} &A_{r3} \cdot B_{c2}\\
        A_{r4} \cdot B_{c1} &A_{r4} \cdot B_{c2}                
      \end{bmatrix} \to
      \begin{bmatrix}
        -27 &-17\\
        5 &1\\
        -53 &-58\\
        15 &36
      \end{bmatrix}
      \end{gathered}
    \]

    \textbf{Power of a Matrix}\\
    If $ A $ is an $ n \times n $ matrix and if $ k $ is a positive integer, then $ A^{l} $ deonotes the product of $ k $ copies of $ A $.
    
    \textbf{The Transpose of a Matrix}\\
    Given an $ m \times n $ matrix $ A $, the transpose of $ A $ is the $ n \times m $ matrix, denoted by $ A^{T}  $, whose columns are formed from the corresponding rows of $ A $.      
    
    \textbf{Ex 8}
    \[
      \begin{gathered}
      A = 
      \begin{bmatrix}
        a &b\\
        c &d
      \end{bmatrix} \qquad 
      A^{T} = \begin{bmatrix}
        a &c\\
        b &d
      \end{bmatrix}\\
      ~\\
      B = \begin{bmatrix}
        -5 &2\\
        1 &-3\\
        0 &4
      \end{bmatrix} \qquad 
      B^{T} = \begin{bmatrix}
        -5 &1 &0\\
        2 &-3 &4
      \end{bmatrix}\\
      ~\\
      C = \begin{bmatrix}
        1 &1 &1 &1\\
        -3 &5 &-2 &7
      \end{bmatrix} \qquad
      C = \begin{bmatrix}
        1 &-3\\
        1 &5\\
        1 &-2\\
        1 &7
      \end{bmatrix}
      \end{gathered}
    \]

    \textbf{Theorem}\\
    Let $ A ~\&~ B $ denote matrices whose sizes are appropiate for the following sums and products.\\
    A) $ (A^{T} )^{T} =A $\\
    B) $ (A+B)^{T} =A^{T}+B^{T}$\\
    C) For any scalar $ r $, $ (rA)^{T} =rA^{2} $\\
    D) $ (AB)^{T}=B^{T}A^{T} $    
    

   I can be treated as 1. 
    
    
\end{document}
