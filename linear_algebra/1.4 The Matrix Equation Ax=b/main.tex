\documentclass{article}
\usepackage[margin=1in]{geometry}
\usepackage{microtype}
\usepackage{setspace}
\usepackage{amsmath}
\usepackage{parskip}
\usepackage{amssymb}
\usepackage{graphicx}

\graphicspath{{../public/}}

\parskip=4ex
\date{}
\author{}

\title{1.4 The Matrix Equation}

\begin{document}
  \maketitle
  \textbf{Introduction}\\
  A fundamental idea in Linear Algebra is to view a linear combination of vectors as the product of a matrix and a vector.

  \textbf{Definition}\\
  If $ A $ is an $ m \times n $ matrix, with columns $ a_{1},...a_{n} $, and if $ x $ is in $ \mathbb{B}^{n}$, then the product of $ A ~\&~ x $, denoted by $ Ax $, is the linear combination of the columns of $ A $ using the corresponding entries in x as weights.
  \[
    Ax=\begin{bmatrix}
      a_{1} &a_{2} &... &a_{n} 
    \end{bmatrix} \begin{bmatrix}
      x_{1}\\
      .\\
      .\\
      .\\
      x_{n}
    \end{bmatrix} = x_{1}a_{1} + x_{2}a_{2} + ...+x_{n} a_{n}    
  \]

  Note that $ Ax $ is defined only if the number of columns of $ A $, $ n $, equals the number of entries in $ x $.
  
  \textbf{Ex 1}
  \[
    \begin{gathered}
    \begin{bmatrix}
      1 &2 &-1\\
      0 &-5 &3
    \end{bmatrix}
    \begin{bmatrix}
      4\\
      3\\
      7
    \end{bmatrix} = 
    4\begin{bmatrix}
      1\\
      0
    \end{bmatrix} +
    3\begin{bmatrix}
      2\\
      -5
    \end{bmatrix} +
    7 \begin{bmatrix}
      -1\\
      3
    \end{bmatrix} =
    \begin{bmatrix}
      3\\
      6
    \end{bmatrix}
    \end{gathered}
  \]

  \textbf{Ex 2}\\
  For $ v_{1},v_{2},v_{3} \in \mathbb{R}^{m}$, write the linear combination $ 3v_{1}-5v_{2}+7v_{3}$ as a matrix times a vector.
  \[
    3v_{1}-5v_{2}+7v_{3}=
    \begin{bmatrix}
      v_{1} &v_{2} &v_{3} 
    \end{bmatrix}
    \begin{bmatrix}
      3\\
      -5\\
      7
    \end{bmatrix}=Ax
  \]

  In section 1.3, we were taught to write a system of linear equations as a vector equation involving a linear combination of vectors. For example, the system
  \[
    \begin{gathered}
    x_{1}+2x_{2}-x_{3}=4\\
    -5x_{2}+3x_{3}=1     
    \end{gathered}
  \]
  is equivalent to

  \[
    x_{1}\begin{bmatrix}
      1\\
      0
    \end{bmatrix} +
    x_{2} \begin{bmatrix}
      2\\
      -5
    \end{bmatrix} +
    x_{3}\begin{bmatrix}
      -1\\
      3
    \end{bmatrix}=
    \begin{bmatrix}
      4\\
      1
    \end{bmatrix}
  \]
  Which can also be rewritten in the $ Ax=b $ form like so

  \[
    \begin{bmatrix}
      1 &2 &-1\\
      0 &-5 &3
    \end{bmatrix}
    \begin{bmatrix}
      x_{1} \\
      x_{2} \\
      x_{3} 
    \end{bmatrix} =
    \begin{bmatrix}
      4\\
      1
    \end{bmatrix}
  \]
  The form $ Ax=b $ is called a matrix equation.

  \textbf{Theorem}\\
  If an $ m \times n $ matrix, $ A $, with columns $ a_{1} ,...,a_{n}  $ and if $ b \in \mathbb{R}^{m}$
  \[
    Ax=b
  \]

  will have the same solution set as the vector equation
  \[
    x_{1} a_{1} +x_{2} a_{2} +...+x_{n} a_{n} =b
  \]

  which will also have the same solution set as the system of linear equations whose augmented matrix is
  \[
    \begin{bmatrix}
      a_{1} &a_{2} &... &a_{n} &b
    \end{bmatrix}
  \]
 
  In Section 1.3, there was an existence question, "Is $ b $ in Span $\{a_{1},...,a_{n}  \}$?" Or sometimes written as "Is $ Ax=b $ consistent?" The question that we will now ask is an extension of the aforementioned existence question, whether the equation $ Ax=b ,~\{b|b\in \mathbb{R}^{m} \}$ holds true or not.

  \textbf{Ex 3}\\
  Let $ A=\begin{bmatrix}
    1 &3 &4\\
    -4 &2 &-6\\
    -3 &-2 &-7
  \end{bmatrix} $ and $ b=\begin{bmatrix}
    b1\\
    b2\\
    b3
  \end{bmatrix} $. Is the equation $ Ax=b $ consistent for all possible $ b1,b2,b3 $?
  \[
    \begin{gathered}
    \begin{bmatrix}
      1 &3 &4b &b1\\
      -4 &2 &-6 &b2\\
      -3 &-2 &-7 &b3
    \end{bmatrix} \to
    \begin{bmatrix}
      1 &3 &4 &b1\\
      0 &14 &10 &b_{2}+4b_{1}\\
      0 &7 &5 &b_{1}-\frac{1}{2} b_{2}+b3    
    \end{bmatrix}
    \end{gathered}
  \]

  The equation $ Ax=b $ is not consistent for every $ b $ because there exists values of $ b $ that will make $ b_{1} - \frac{1}{2}b_{2}  + b_{3}  \neq 0 $. Hence the entries in $ b $ must satisfy
  \[
    b_{1} -\frac{1}{2} b_{2} +b_{3}=0 
  \]
  Which is also the equation of a plane through the origin in $ \mathbb{R}^{3}$. Hence the plane is the set of all linear combinations of the three columns belong to $ A $

  To extend, the equation $ Ax=b $ failed to be consistent for all values of $ b $ because the echelon form of $ A $ has a row of zeros. 

  \textbf{Theorem}\\
  Let $ A $ be an $ m \times n $ matrix. The following statements are logically equivalent meaning that they are all either true statements otherwise they are all false.

A) For each $ b \in \mathbb{R}^{m}$, the equation $ Ax=b $ has a solution.\\
  B) Each $ b \in \mathbb{R}^{m}$ is a linear combination of the columns of $ A $.\\
  C) The columns of $ A $ span $ \mathbb{R}^{m} $.\\
  D) $ A $ has a pivot position in every row. 

\end{document}
