\documentclass{article}
\usepackage[margin=1in]{geometry}
\usepackage{microtype}
\usepackage{setspace}
\usepackage{amsmath}
\usepackage{parskip}
\usepackage{amssymb}

\parskip=4ex
\date{}
\author{}

\title{5.2 The Characteristic Equation}

\begin{document}
  \maketitle
  \textbf{Introduction}\\
  Useful information about the eigenvalues of a square matrix $ A $ is encoded by a special scalar equation known as the characteristic equation of $ A $.

  We must find all scalars $ \lambda $ such that the equation above has a nontrivial solution. By the Invertible Matrix Theorem, this problem is equivalent to finding all $ \lambda $ such that the matrix $ A - \lambda I $ is not invertible.

  \textbf{Ex 1}\\
  Find the egienvalues of $ A=\begin{bmatrix}
    2 &3\\
    3 &-6
  \end{bmatrix} $.
  \[
    \begin{gathered}
      (A-\lambda I)x=0
    \end{gathered}
  \]

  \[
    \begin{gathered}
     A - \lambda I = \begin{bmatrix}
      2 &3\\
      3 &-6
    \end{bmatrix} -
    \begin{bmatrix}
      \lambda &0\\
      0 &\lambda
    \end{bmatrix} =
    \begin{bmatrix}
      2 -\lambda &3\\
      3 &-6-\lambda
    \end{bmatrix}\\
    ~\\
    \text{det}(A-\lambda I)=0
    \end{gathered}
  \]

  Then $ \lambda=3 $ or $ \lambda=-7 $, so the eigenvalues of $ A $ are $ 3 ~\&~ -7 $.

  \textbf{Ex 2}\\
  Find the egienvalues of $ A=\begin{bmatrix}
    2 &3\\
    3 &-6
  \end{bmatrix} $.
  \[
    \begin{gathered}
      (A-\lambda I)x=0
    \end{gathered}
  \]

  We must find all scalars $ \lambda $ such that the equation above has a nontrivial solution. By the Invertible Matrix Theorem, this problem is equivalent to finding all $ \lambda $ such that the matrix $ A - \lambda I $ is not invertible.
  \[
    A - \lambda I = \begin{bmatrix}
      2 &3\\
      3 &-6
    \end{bmatrix} -
    \begin{bmatrix}
      \lambda &0\\
      0 &\lambda
    \end{bmatrix} =
    \begin{bmatrix}
      2 -\lambda &3\\
      3 &-6-\lambda
    \end{bmatrix}
  \]

  The matrix fails to be invertible when its determinant is zero. So the eigenvalues of $ A $ are the solutions to the equation
  \[
    \begin{gathered}
    \text{det}(A-\lambda I) = \begin{vmatrix}
    2-\lambda &3\\
    3 &-6-\lambda
    \end{vmatrix}=0\\
    ~\\
    \text{det}(A-\lambda I)=(2-\lambda)(-6-\lambda)-(3)(3)\\
    -12+6\lambda-2\lambda+\lambda^{2}-9\\
    \lambda^{2}+4\lambda-21\\
    (\lambda-3)(\lambda+7)
    \end{gathered}
  \]

  So if det$ (A-\lambda I)=0 $, then $ \lambda=3 $ or $ \lambda=-7$, then the eigenvalues of $ A $ are $ 3 ~\&~ -7 $.
  
  \textbf{The Invertible Matrix Theorem (continued)}\\
  Let $ A $ be an $ n \times n $ matrix. Then $ A $ is invertible if and only if

  r) The number $ 0 $ is not an eigenvalue of $ A $.
  
  \textbf{The Characteristic Equation}\\
  A scalar $ \lambda $ is an eigenvalue of an $ n \times n $ matrix $ A $ if and only if $ \lambda $ satisfies the characteristic equation
  \[
    \text{det}(A- \lambda I)=0
  \]

  The eigenvalue $ 5 $ is said to have multiplicity 2 because $ (5-\lambda) $ occurs twice as a factor of the characteristic polynomial. The algebraic multiplicity of an eigenvalue $ \lambda $ is its multiplicity as a root of the characteristic equation.

  \textbf{Theorem 5.4}\\
  If $ n \times n $ matrices $ A ~\&~ B $ are similar, then they have the same characteristic polynomial and hence the same eigenvalues (with the same multiplicities)
\end{document}
