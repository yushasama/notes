\documentclass{article}
\usepackage[margin=1in]{geometry}
\usepackage{microtype}
\usepackage{setspace}
\usepackage{amsmath}
\usepackage{parskip}
\usepackage{amssymb}

\parskip=4ex
\date{}
\author{}

\title{5.2 The Characteristic Equation}

\begin{document}
  \maketitle
  When solving for scalars $ \lambda $ such that the matrix equation
  \[
    (A-\lambda I)x=0
  \]

  has a nontrivial solution. Meaning that by the Invertible Matrix Theorem, we find that this problem is equivalent to finding all $ \lambda $ such that matrix $ A - \lambda I $ is not invertible.

  The matrix fails to be invertible precisely when its determinant is zero. So the eigenvalues of $ A $ are the solutions of the equation
  \[
    \text{det}(A-\lambda I) = 0
  \]

  \textbf{Theorem 5.3 Properties of Determinants}\\
  Let $ A ~\&~ B $ be $ n \times n $ matrices.

  a) $ A $ is invertible if and only if det $ A\neq0 $
  
  b) det $ AB =$ (det $ A $ )(det $ B $ )

  c) det $ A^{T} =$ det $ A $

  d) If $ A $ is triangular, then det $ A $ is the product of the entries on the main diagonal of $ A $.

  e) A row replacement operation on $ A $ does not change the determinant. A row interchange changes the sign of the determinant. A row scaling also scales the determinant by the same scalar factor.

  \textbf{The Invertible Matrix THeorem}\\
  Let $ A $ be an $ n \times n $ matrix. Then $ A $ is invertible if and only if
  
  r) The number $ 0 $ is not an eigenvalue of $ A $.

  \textbf{The Characteristic Equation}\\
  The scalar equation det $ (A_ \lambda I)=0 $ is called the characteristic equation of $ A $.

  A scalar $ \lambda $ is an eigenvalue of an $ n \times n $ matrix $ A $ if and only if $ \lambda $ satisfies the characteristic equation
  \[
    \text{det}(A-\lambda I)=0
  \]

  \textbf{Ex 1}\\
  Find the characteristic equation of 
  \[
    A = 
    \begin{bmatrix}
      5 &-2 &6 &-1\\
      0 &3 &-8 &0\\
      0 &0 &5 &4\\
      0 &0 &0 &1
    \end{bmatrix}
  \]

  \[
    \begin{gathered}
    \text{det}(A_\lambda I)=
    \begin{vmatrix}
    5-\lambda &-2 &6 &-1\\
    0 &3-\lambda &-8 &0\\
    0 &0 &5-\lambda &4\\
    0 &0 &0 &1-\lambda
    \end{vmatrix}
    \end{gathered}\\
    (5-\lambda)(3-\lambda)(5-\lambda)(1-\lambda)
  \]

  The characteristic equation is 
  \[
    (5-\lambda)^{2}(3-\lambda)(1-\lambda)=0
  \]

  Expanding the product, we can also write
  \[
    \lambda^{4}-14\lambda^{3}+68\lambda^{2}-130\lambda+75=0
  \]

  So the matrix $ A $ has the eigenvalues $ 5,3,1 $, with $ 5 $ having multiplicity 2.
  
  


  


  
  


  
  
  
\end{document}
