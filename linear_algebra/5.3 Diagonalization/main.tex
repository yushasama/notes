\documentclass{article}
\usepackage[margin=1in]{geometry}
\usepackage{microtype}
\usepackage{setspace}
\usepackage{amsmath}
\usepackage{parskip}
\usepackage{amssymb}

\parskip=4ex
\date{}
\author{}

\title{5.3 Diagonalization}

\begin{document}
    \maketitle
    \textbf{Ex 1}\\
    If $ D=\begin{bmatrix}
        5 &0\\
        0 &3
    \end{bmatrix} $, then
    $ D^{2}=\begin{bmatrix}
        5 &0\\
        0 &3
    \end{bmatrix}
    \begin{bmatrix}
        5 &0\\
        0 &3
    \end{bmatrix}=
    \begin{bmatrix}
        5^{2} &0\\
        0 &3^{2}  
    \end{bmatrix}
    $

    and
    \[
        D^{3}=DD^{2}=
        \begin{bmatrix}
            5 &0\\
            0 &3
        \end{bmatrix}
        \begin{bmatrix}
            5^{2}  &0\\
            0 &3^{2} 
        \end{bmatrix}=
        \begin{bmatrix}
            5^{3} &0\\
            0 &3^{3}  
        \end{bmatrix}
    \]

    In general,
    \[
        D^{k}=
        \begin{bmatrix}
            5^{k} &0\\
            0 &3^{k}  
        \end{bmatrix}, \qquad \text{for }k \ge 1
    \]

    If $ A=PDP^{-1} $, for some invertible $ P $ and diagonal $ P $, then $ A^{k} $ is also easy to compute.

    \textbf{Ex 2}\\
    Let $ A=\begin{bmatrix}
        7 &2\\
        -4 &1
    \end{bmatrix} $. Find a formula for $ A^{k} $, given that $ A=PDP^{-1} $, where
    \[
        P=\begin{bmatrix}
            1 &1\\
            -1 &-2
        \end{bmatrix} \qquad ~\&~ \qquad 
        D=\begin{bmatrix}
            5 &0\\
            0 &3
        \end{bmatrix}
    \]

    \[
        \begin{gathered}
        P^{-1}=\begin{bmatrix}
            2 &1\\
            -1 &-1
        \end{bmatrix}\\
        ~\\
        A^{2}=(PDP^{-1})(PDP^{-1})=PD(P^{-1}P)DP^{-1}=PDDP^{-1}, \qquad P^{-1}P=I\\
        PD^{2}P^{-1}=\begin{bmatrix}
            1 &1\\
            -1 &-2
        \end{bmatrix}
        \begin{bmatrix}
            5^{2} &0\\
            0 &3^{2}  
        \end{bmatrix}
        \begin{bmatrix}
            2 &1\\
            -1 &-1
        \end{bmatrix}\\
        ~\\
        A^{3}=(PDP^{-1})(PDP^{-1})(PDP^{-1})=PD(P^{-1}PPP^{-1}P^{-1}P)D^{2} =PD^{3}P^{-1}
        \end{gathered}
    \]

    So in general, for $ k \ge 1 $
    \[
        \begin{gathered}
        A^{K}=PD^{K}P^{-1}=
        \begin{bmatrix}
            1 &1\\
            -2 &-2
        \end{bmatrix}
        \begin{bmatrix}
            5^{k} &0\\
            0 &3^{k}  
        \end{bmatrix}
        \begin{bmatrix}
            2 &1\\
            -1 &-1
        \end{bmatrix}
        \end{gathered}\\
        \begin{bmatrix}
            2\cdot 5^{k}-3^{k} &5^{k}-3^{k}\\
            2 \cdot 3^{k}-2\cdot 5^{k} &2\cdot 3^{k}-5^{k}       
        \end{bmatrix}
    \]

    \textbf{The Diagonalization Theorem}\\
    An $ n \times n $ matrix $ A $ is diagonalizable if and only if $ A $ has $ n $ linearly indepedent vectors.

    $ A=PDP^{-1} $, with $ D $  
\end{document}
