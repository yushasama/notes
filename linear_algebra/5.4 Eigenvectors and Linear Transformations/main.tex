\documentclass{article}
\usepackage[margin=1in]{geometry}
\usepackage{microtype}
\usepackage{setspace}
\usepackage{amsmath}
\usepackage{parskip}
\usepackage{amssymb}
\usepackage{graphicx}

\graphicspath{{../public/}}

\parskip=4ex
\date{}
\author{}

\title{5.4 Eigenvectors and Linear Transformations}

\begin{document}
  \maketitle
  \textbf{Definition}\\
  Let $ V $ be a vector space. An eigenvector of a linear transformation $ T:V\to V $ is a nonzero vector in $ V $ suc that $ T(x)=\lambda x $ for some scalar $ \lambda $. A scalar $ \lambda $ is an eigenvalue of $ T $ if there is a nontrivial solution to $ T(x)=\lambda x $. Such an $ x $ is known as an eigenvector corresponding to $ \lambda $.

  \textbf{The Matrix of a Linear Transformation}\\
  Let $ V $ be an $ n $ dimensional vector space and let $ T $ be any linear trasnformation from $ V $ to $ V $. To associate a matrix with $ T $, choose any basis $ \mathcal{B} $ for $ V $. Given any vector $ x $, the coordinate vector $ \begin{bmatrix}
    x
  \end{bmatrix}_\mathcal{B} $ is in $ \mathbb{R}^{n} $, as is the coordinate vector of its image, $ \begin{bmatrix}
    T(x)
  \end{bmatrix}_\mathcal{B} $.

  Let $ \{ b_1,...,b_n \} $ be the basis $ \mathcal{B} $ for $ V $. If $ x=r_1b_1+...+r_nb_n $, then
  \[
    \begin{bmatrix}
    x
  \end{bmatrix}_\mathcal{B} =
  \begin{bmatrix}
    r_1\\
    \vdots\\
    r_n
  \end{bmatrix}
  \]
  
  And
  \[
    T(x)=T(r_1b_1+...+r_nb_n)=r_1T(b_1)+...+r_nT(b_n)
  \]

  because $ T $ is linear. Since the coordinate mapping from $ B $ to $ \mathbb{R}^{n} $ is linear, we get
  \[
    \begin{gathered}
     T(x)=T(r_1b_1+...+r_nb_n)=r_1T(b_1)+...+r_nT(b_n)\\
     \begin{bmatrix}
       T(x)
     \end{bmatrix}_\mathcal{B}=r_1
     \begin{bmatrix}
       T(b_1)
     \end{bmatrix}_\mathcal{B}+...+
     r_n\begin{bmatrix}
       T(b_n)
     \end{bmatrix}_\mathcal{B}
    \end{gathered}
  \]
  
  Since $ \mathcal{B} $ coordinate vectors are in $ \mathbb{R}^{n} $, the vector equation can be written as a matrix equation, namely
  \[
    \begin{bmatrix}
      T(x)
    \end{bmatrix}_\mathcal{B}=M
    \begin{bmatrix}
      x
    \end{bmatrix}_\mathcal{B}
  \]

  where 
  \[
    M=\begin{bmatrix}
      \begin{bmatrix}
        T(b_1)
      \end{bmatrix}_\mathcal{B} &\begin{bmatrix}
        T(b_2)
      \end{bmatrix}_\mathcal{B} &... 
      &\begin{bmatrix}
        T(b_n)
      \end{bmatrix}_\mathcal{B}
    \end{bmatrix}
  \]
 
  The matrix $ M $ is a matrix representation of $ T $, called the matrix for $ T $ relative to the basis $ \mathcal{B} $ and denoted by $ \begin{bmatrix}
    T
  \end{bmatrix}_\mathcal{B} $.
  
  \textbf{Ex 1}\\
  Suppouse $ \mathcal{B}=\{ b_1,b_2 \} $ is a basis for $ V $. Let $ T: V\to V $ be a linear transformation with the property that
  \[
    T(b_1)=3b_1-2b_2 \qquad T(b_2)=4b_1+7b_2
  \]

  Find the matrix $ M $ for $ T $ relative to $ \mathcal{B} $.

  \[
    \begin{gathered}
    \begin{bmatrix}
      T(b_1)
    \end{bmatrix}_\mathcal{B}=
    \begin{bmatrix}
      3\\
      -2
    \end{bmatrix} \qquad 
    \begin{bmatrix}
      T(b_2)
    \end{bmatrix}_\mathcal{B}=
    \begin{bmatrix}
      4\\
      7
    \end{bmatrix}\\
    ~\\
    M=\begin{bmatrix}
      3 &4\\
      -2 &7
    \end{bmatrix}
    \end{gathered}
  \]

  \textbf{Ex 2}\\
  The mapping $ T:\mathbb{P}_2 \to \mathbb{P}_2 $ defined by
  \[
    T(a_0+a_1t+a_2t^{2})=a_1+2a_2t
  \]

  $ T $ can be regonized as the differentiation operator

  a) Find the $ \mathcal{B} $ matrix for $ T $, when $ B $ is the basis $ \{ 1,t,t^{2} \} $ 

  Firstly, compute the images of the basis vectors
  \[
    \begin{gathered}
    T(1)=0\\
    T(t)=1\\
    T(t^{2})=2t
    \end{gathered}
  \]

  So 
  \[
    \begin{bmatrix}
      T(1)
    \end{bmatrix}_\mathcal{B}=
    \begin{bmatrix}
      0\\
      0\\
      0
    \end{bmatrix} \qquad
    \begin{bmatrix}
      T(t)
    \end{bmatrix}_\mathcal{B}=\begin{bmatrix}
      1\\
      0\\
      0
    \end{bmatrix} \qquad
    \begin{bmatrix}
      T(t^{2})
    \end{bmatrix}=
    \begin{bmatrix}
      0\\
      0\\
      0
    \end{bmatrix}\\
    ~\\
    \begin{bmatrix}
      T
    \end{bmatrix}_\mathcal{B}=
    \begin{bmatrix}
      0 &1 &0\\
      0 &0 &2\\
      0 &0 &0
    \end{bmatrix}
  \]

  b) Verify that $\begin{bmatrix}
    T(p)
  \end{bmatrix}_\mathcal{B} =\begin{bmatrix}
    T
  \end{bmatrix}_\mathcal{B}\begin{bmatrix}
    p
  \end{bmatrix}_\mathcal{B} $ for each $ p $ in $ \mathbb{P}_2$.

  For a general $ p(t)=a_0+a_1t+a_2t^{} $ 
  \[
    \begin{gathered}
    \begin{bmatrix}
      T(p)
    \end{bmatrix}_\mathcal{B}=\begin{bmatrix}
      a_1+2a_2t
    \end{bmatrix}_\mathcal{B}=\begin{bmatrix}
      a_1\\
      2a_2\\
      0
    \end{bmatrix}=
    \begin{bmatrix}
      0 &1 &0\\
      0 &0 &2\\
      0 &0 &0
    \end{bmatrix}
    \begin{bmatrix}
      a_0\\
      a_1\\
      a_2
    \end{bmatrix}=
    \begin{bmatrix}
      T
    \end{bmatrix}_\mathcal{B}
    \begin{bmatrix}
      p
    \end{bmatrix}_\mathcal{B}
    \end{gathered}
  \]

  \textbf{Theorem 5.8 Diaognal Matrix Representation}\\
  Suppouse $ A=PDP^{-1} $ ,where $ D $ is a diagonal $ n\times n $ matrix. If $ \mathcal{B} $is the basis for $ \mathbb{R}^{n} $ formed from the columns of $ P $, then $ D $  is the $ \mathcal{B} $ matrix for the transformation $ x\mapsto Ax $. 

  \textbf{Ex 4}\\
  Define $ T:\mathbb{R}^{2}\to \mathbb{R}^{2} $ by $ T(x)=Ax $, where $ A=\begin{bmatrix}
    7 &2\\
    -4 &1
  \end{bmatrix} $. Find a basis $ \mathcal{B} $ for $ \mathbb{R}^{2} $ with the property that the $ \mathcal{B} $ matrix for $ T $ is a diagonal matrix.

  We know that $ A=PDP^{-1} $, where
  \[
    P=\begin{bmatrix}
      1 &1\\
      -1 &-2
    \end{bmatrix} \qquad 
    D=\begin{bmatrix}
      5 &0\\
      0 &3
    \end{bmatrix}
  \]

  The columns of $ P $ are $ b_1 ~\&~ b_2 $, which are the eigenvectors of $ A $. $ D $ is the $ \mathcal{B} $ matrix for $ T $ when $ \mathcal{B}=\{ b_1,b_2 \} $. 
\end{document}
