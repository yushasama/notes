\documentclass{article}
\usepackage[margin=1in]{geometry}
\usepackage{microtype}
\usepackage{setspace}
\usepackage{amsmath}
\usepackage{parskip}
\usepackage{amssymb}
\usepackage{graphicx}

\graphicspath{{../public/}}

\parskip=4ex
\date{}
\author{}

\title{4.5 The Dimension of a Vector Space}

\begin{document}
  \maketitle
  \textbf{Theorem 4.10}\\
  If a vector space $ V $ has a basis $ \mathcal{B}=\{ b_1,...,b_n \} $, then any set in $ V $ containing more than $ n $ vectors must be linearly dependent.

  \textbf{Proof}\\
  Let $ \{ u1,...,u_p \} $ be a set in $ V $ with more than $ n $ vectors. The coordinate vectors $ \begin{bmatrix}
    u_1
  \end{bmatrix}_\mathcal{B},...,\begin{bmatrix}
    u_p
  \end{bmatrix}_\mathcal{B} $ form a linearly dependent set in $ \mathbb{R}^{n} $, because there are more vectors $ (p) $ than entries $ (n) $ in each vector. So there exists scalars $ c_1,...,c_p $, not all zero such that
  \[
    c_1\begin{bmatrix}
      u_1
    \end{bmatrix}_\mathcal{B}+...+c_p\begin{bmatrix}
      u_p
    \end{bmatrix}_\mathcal{B}=\begin{bmatrix}
      0\\
      \vdots\\
      0
    \end{bmatrix}
  \]

  Since the coordinate mapping is a linear transformation,
  \[
    \begin{bmatrix}
      c_1u_1+...+c_pu_p
    \end{bmatrix}_\mathcal{B}=\begin{bmatrix}
      0\\
      \vdots\\
      0
    \end{bmatrix}
  \]

  The zero vector displays the $ n $ weights needed to build the vectors $ c_1u_1+...+c_pu_p $ from the basis vectors in $ \mathcal{B} $. That is, $ c_1u_1+...+c_pu_p=0b_1+...+0b_n=0 $. Then since $ c_i $ are not all zero, $ \{ u_1,...,u_p \} $ is linearly dependent.

  Meaning that Theorem 4.10 implies that if a vector space $ V $ has a basis $ \mathcal{B}=\{ b_1,...,b_n \} $, then each linearly independent set in $ V $ has no more than $ n $ vectors.

  \textbf{Theorem 4.11}\\
  If a vector space $ V $ has a basis of $ n $ vectors, then every basis of $ V $ must consistent of exactly $ n $ vectors. 
  
  \textbf{Definition}\\
  If a vector space $ V $ is spanned by a finite set, then $ V $ is said to be finite-dimensional, and the dimension of $ V $, written as dim $ V $, is the number of vectors in a basis for $ V $. The dimension of the zero vector space $ \{ 0 \} $ is defined to be zero. If $ V $ is not spanned by a finite set, then $ V $ is said to be infinite-dimensional.

  \textbf{Ex 1}\\
  The standard basis for $ \mathbb{R}^{n} $ contains $ n $ vectors, so dim $ \mathbb{R}^{n}=n $. The standard polynomial basis $ \{ 1,t,t^{2} \} $ shows that dim $ \mathbb{P}_2=3 $. In general, dim $ \mathbb{P}_n=n+1 $. The space $ \mathbb{P} $ of all polynomials is infinite-dimensional. 
 
  \textbf{Ex 2}\\
  Let $ H= $ Span $ \{ v_1,v_2 \} $, where $ v_1=\begin{bmatrix}
    3\\
    6\\
    2
  \end{bmatrix} $ and $ v_2=\begin{bmatrix}
    -1\\
    0\\
    1
  \end{bmatrix} $. A basis for $ H $ is $ \{ v_1,v_2 \} $, since $ v_1 ~\&~ v_2 $ are not multiples and hence are linearly independent. Thus dim $ H =2$. 

  \textbf{Ex 3}\\
  Find the dimension of the subspace
  \[
    H = \text{\huge{\{}} \begin{bmatrix}
      a-3b+6c\\
      5a+4d\\
      b-2c-d\\
      5d
    \end{bmatrix}: a,b,c,d \in \mathbb{R} \text{\huge{\}}}
  \]

  $ H $ is the set of all linear combinations of the vectors
  \[
    v_1=\begin{bmatrix}
      1\\
      5\\
      0\\
      0
    \end{bmatrix}, \qquad v_2=\begin{bmatrix}
      -3\\
      0\\
      1\\
      0
    \end{bmatrix}, \qquad v_3=\begin{bmatrix}
      6\\
      0\\
      -2\\
      0
    \end{bmatrix}, \qquad v_4=\begin{bmatrix}
      0\\
      4\\
      -1\\
      5
    \end{bmatrix}
  \]
  
  We can see that $ v_3 $ is a multiple of $ v_2 $. So by the Spanning Set Theorem, $ v_3 $ can be discarded and we would stil have a set that spans $ H $. The 3 other vectors are linearly independent so $ \{ v_1,v_2,v_4 \} $ is linearly independent and hence it is a basis for $ H $. Thus
  \[
    \boxed{\text{dim }H=3}
  \]

  \textbf{Subspaces of a Finite-Dimensional Space}\\
  The next theorem is a natural counterpart to the Spanning Set Theorem.

  \textbf{Theorem 4.12}\\
  Let $ H $ be a subspace of a finite-dimensional vector space $ V $. Any linearly independent set in $ H $ can be expanded, if necessary, to a basis for $ H $. Also, $ H $ is finite-dimensional and 
  \[
    \text{dim }H\le \text{dim }V
  \]
 
  \textbf{Theorem 4.13 The Basis Theorem}\\
  Let $ V $ be a p-dimensional vector space, $ b \ge 1 $. Any linearly independent set of exactly $ p $ elements in $ V $ is automatically a basis for $ V $. Any set of exactly $ p $ elements that spans $ V $ is automatically a basis for $ V $.
  
  \textbf{The Dimensions of Nul A, Col A, ~\&~ Row A}\\
  Since the dimensions of the null space and column space of an $ m \times n $ matrix are referred to frequently, they have specific names:

  \textbf{Definition}\\
  The rank of an $ m \times n $ matrix $ A $ is the dimension of the column space and the nullity of $ A $ is the dimension of the null space.
  
  The rank of an $ m \times n $ matrix $ A $ is the number of pivot columns and the nullity of $ A $ is the number of free variables. Since the dimension of the row space is the number of pivot rows, it is also equal to the rank of $ A $.

  \textbf{Theorem 4.14 The Rank Theorem}\\
  The dimensions of the column space and the null space of an $ m \times n $ matrix $ A $ satisfy the equation
  \[
    \text{rank }A+ \text{nullity }A= \text{number of columns in }A
  \]

  \textbf{Ex 5}\\
  Find the nullity and rank of
  \[
    \begin{gathered}
    A = \begin{bmatrix}
      -3 &6 &-1 &1 &-7\\
      1 &-2 &2 &3 &-1\\
      2 &-4 &5 &8 &-4
    \end{bmatrix}\\
    ~\\
    Ax=0, \text{ after row reduction}
    \begin{bmatrix}
      1 &-2 &2 &3 &-1 &0\\
      0 &0 1 &2 &-2 &0\\
      0 &0 0 &0 &0 &0
    \end{bmatrix}
    \end{gathered}
  \]

  There are three free variables,ll.run(2133)
  .$ x_2,x_4, ~\&~ x_5 $. Hence the nullity of $ A $ is 3. Also the rank of A is $ 2 $ because $ A $ has two pivot columns.

  \textbf{Ex 6}\\
  a) If $ A $ is a $  7 \times 9 $ matrix with nullity 2, what is the rank of a?

  Since $ A $ has 9 columns, rank $ A $ + 2 =9, rank $ A =7$.

  b) Could a $ 6 \times 9 $ matrix have nullity 2?

  The columns of the matrix are vectors in $ \mathbb{R}^{6} $, thus the dimension of the columns cannot exceed 6, that is the rank cannot exceed 6. So 6 + nulllity $ A =9$ to be true, the nullity has to be $ 3 $.

  \textbf{The Invertible Matrix Theorem (continued)}\\
  Let $ A $ be an $ n \times n $ matrix. Then the following statements are each equivalent to the statement that $ A $ is an invertible matrix.

  m) The columns of $ A $ form a basis of $ \mathbb{R}^{n} $.
  n) Col $ A =\mathbb{R}^{n}$
  o) rank $ A=n $
  p) nullity $ A =0$
  q) Nul $ A =\{ 0 \}$ 
  

\end{document}
