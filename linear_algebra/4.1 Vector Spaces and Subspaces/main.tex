\documentclass{article}
\usepackage[margin=1in]{geometry}
\usepackage{microtype}
\usepackage{setspace}
\usepackage{amsmath}
\usepackage{parskip}
\usepackage{amssymb}
\usepackage{graphicx}

\graphicspath{{../public/}}

\parskip=4ex
\date{}
\author{}

\title{4.1 Vector Spaces and Subspaces}

\begin{document}
    \maketitle
    \textbf{Definition}\\
    a vector space is a nonempty set $ V $ of objects, called vectors, on which are defined two operations, called addition and multiplication by scalars (real numbers), subject to the ten axios below. The axioms must hold for all vectors $ u,v, ~\&~ w $ in $ V $ and for all scalars $ c ~\&~ d $.
    1) The sum of $ u ~\&~ v $, $ u+v $ is in $ V $.
    2) $ u + v = v+ u $.
    3) $ (u+v) + w = u + (v+w) $.
    4) There is a zero vector $ 0 $ in $ V $ such that $ u + 0 =u $.
    5) For each $ u $ in $ V $, there is a vector $ -u $ in $ V $ such that $ u+(-u)=0 $. 
    6) The scalar multiple, denoted by $ cu $, is in $ V $.
    7) $ c(u+v)=cu+cv $.
    8) $ (c+d)u=cu=du $.
    9) $ c(du)=(cd)u $.
    10) $ 1u=u $ 

    \textbf{Ex 4}\\
    For $ n \le 0 $, the set $ \mathbb{P}_{n} $ of polynomials of degree at most $ n $ consists of the all polynomials of the form
    \[
        p(t)=a_{0}+a_{1}t+a_{2}t^{2}+...+a_{n}t^{n}      
    \]
    Where the coefficients $ a_{0},...,a_{n} $ and the variable $ t $ are real numbers. If $ q(t)=b_{0}+b_{1}t+...+b_{n}t^{n} $, then the sum $ p+q $ is defined by
    \[
      (p+q)(t)=p(t)+q(t)=(a_{0}+b_{0})t+(a_{1}+b_{1})t+...+(a_{n}+b_{n})t^{n} 
    \]

    The scalar multiple $ cp $ is the polynomial defined by
    \[
      (cp)(t)=cp(t)=ca_{0}+(ca_{1})t+...+(ca_{n})t^{n}  
    \]

    These definitions satisfy Axioms $ 1 ~\&~ 6 $ because $ p+q ~\&~ cp $ are polynomials of degrees less than or equal to $ n $. We assume the other Axios $ 2,3,~\&~ 7-10 $ follow from the properties of real numbers. The zero polynomial acts as the zero vector in Axiom $ 4 $. $ (-1)p $ acts as the negative of $ p $ so Axiom $ 5 $ is satisfied. Thus $ \mathbb{P}_{n} $ is a vector space.

    \textbf{Subspaces}\\
    A subspace of a vector space $ V $ is a subset $ H $ of $ V $ that has the three properties
    a) The zero vector of $ V $ is in $ H^{2} $
    b) $ H $ is closed under vector addition. Meaning for each $ u+v $ in $ H $, the sum $ u+v $ is in $ H $.
    c) $ H $ is closed under multiplication by scalars. That is, for each vector $ u $ and each scalar $ c $, the vector $ cu $ is in $ H $.

    If these three vector space axioms hold true, then the other seven axioms from the previous definition are automatically satisfied. So every subspace is a vector space.

    \textbf{Ex 9}\\
    The vector space $ \mathbb{R}^{2} $ is not a subspace of $ \mathbb{R}^{3} $ because $ \mathbb{R}^{2} $ is not even a subset of $ \mathbb{R}^{3} $. As the vectors in $ \mathbb{R}^{3} $ all have three entries, where as the vectors in $ \mathbb{R}^{2} $ have only two.

    The set
    \[
      H=\{ \begin{bmatrix}
          s\\
          t\\
          0
      \end{bmatrix}: s,t \in \mathbb{R} \}
    \]
    is a subset of $ \mathbb{R}^{3} $ that looks and acts like $ \mathbb{R}^{2} $, although it is distint from $ \mathbb{R}^{2}  $. Show that $ H $ is a subspace of $ \mathbb{R}^{3} $.

    The zero vector is in $ H $ and $ H $ is closed under vector addition and scalar multiplication. This is due to operations on vectors in $ H $ always produce vectors whose third entries are zero and so belong to $ H $. Hence, $ H $ is a subspace of $ \mathbb{R}^{3} $.

  \textbf{Ex 10}\\
  A plane $ \mathbb{R}^{3} $ that is not through the origin is not a subspace of $ \mathbb{R}^{3} $ because the plane does not contian the zero vector of $ \mathbb{R}^{3} $. The same can be said for a line in $ \mathbb{R}^{2} $ not through the origin, namely the line is not a subspace of $ \mathbb{R}^{2} $.

  \textbf{Ex 11}\\
  Given $ v_1 ~\&~ v_2 $ in a vector space $ V $, let $ H=\text{Span}\{ v_1 ,v_2\} $. Show that $ H $ is a subspace of $ V $.
  \[
    \begin{gathered}
    0=0v_1 + 0v_2\\
    u=s_1v_1 + s_2 v_2 ~\&~ w=t_1v_1 + t_2v_2\\
    u+w = (s_1v_1+s_2v_2) + (t_1v_1+t_2v_2)\\
    (s_1+t_1)v_1 + (s_2+t_2)v_2
    \end{gathered}
  \]

  As the vector $ u+w $ can be written in the form of $ a_0v_1 + a_1v_1 $, $ u+w $ is i $ H $.
  \[
    cu=c(s_1v_1+s_2v_2)=(cs_1)v_1+(cs_2)v_2
  \]

  Because the zero vector is in $ H $ as well as $ H $ being closed under scalar addition and vector addition. Thus $ H $ is a subspace of $ V $.
\end{document}
