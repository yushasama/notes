\documentclass{article}
\usepackage[margin=1in]{geometry}
\usepackage{microtype}
\usepackage{setspace}
\usepackage{amsmath}
\usepackage{parskip}
\usepackage{amssymb}

\parskip=4ex
\date{}
\author{}

\title{}

\begin{document}
    \maketitle

    When a basis $ \mathcal{B} $ is chosen for an n-dimensional vector space $ V $, the associated coordinate mapping onto $ \mathbb{R}^{n} $ provides a coordinate system for $ V $. Each $ x $ in $ V $ is identified uniquely bt its $ \mathcal{B} $ coordinate vector $ \begin{bmatrix}
        x
    \end{bmatrix}_{\mathcal{B}} $.

    \textbf{Ex 1}\\
  Consider two bases $ \mathcal{B} = \{ b_{1},b_{2} \} $ and $ C=\{ c_{1},c_{2} \} $ for a vector space $ V $, such that
  \[
      b_{1}=4c_{1}+c_{2} \qquad b_{2} = -6c_{1}+c_{2}      
  \]

  Suppouse that
  \[
      x=3b_{1}+b_{2}  
  \]

  That is, suppouse $ \begin{bmatrix}
      x
  \end{bmatrix}_{\mathcal{B}} = \begin{bmatrix}
      3\\
      1
  \end{bmatrix} $. Find $ \begin{bmatrix}
      x
  \end{bmatrix}_{\mathcal{C}}  $.

  Apply the coordinate mapping determined by $ C $.
  \[
      \begin{gathered}
      \begin{bmatrix}
          x
      \end{bmatrix}_{C} = \begin{bmatrix}
          3b_{1}+b_{2}  
      \end{bmatrix}_{C}\\
      \begin{bmatrix}
          x
      \end{bmatrix}_{C}= 
      \begin{bmatrix}
          \begin{bmatrix}
              b_{1} 
          \end{bmatrix}_{C}

      &\begin{bmatrix}
          b_{2} 
      \end{bmatrix}_{C}
      \end{bmatrix}
      \begin{bmatrix}
          3\\
          1
      \end{bmatrix}\\
      \begin{bmatrix}
          x
      \end{bmatrix}_{C} =
      \begin{bmatrix}
          4 &-6\\
          1 &1
      \end{bmatrix}
      \begin{bmatrix}
          3\\
          1
      \end{bmatrix}=
      \boxed{\begin{bmatrix}
          6\\
          4
      \end{bmatrix}}
      \end{gathered}
  \]
  
  \textbf{Theorem 5}\\
  Let $ \mathcal{B} = \{ b_{1},...,b_{n} \} ~\&~ C=\{ c_{1},...,c_{n} \} $ be bases for a vector space $ V $. There is a unique $ n \times n $ matrix $ P $ sich that
  \[
      \begin{bmatrix}
          x
      \end{bmatrix}_{C}= P_{C \leftarrow B} \begin{bmatrix}
          x
      \end{bmatrix}_{B}   
  \]
  and $ P_{C \leftarrow B}  $ is given by $ P_{C \leftarrow P}  = \begin{bmatrix}
      \begin{bmatrix}
          b_{1} 
      \end{bmatrix}_{C} &\begin{bmatrix}
          b_{2} 
      \end{bmatrix}_{C} &...
      &\begin{bmatrix}
          b_{n} 
      \end{bmatrix}_{C} 
  \end{bmatrix} $ 
  
\end{document}
