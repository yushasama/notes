\documentclass{article}
\usepackage[margin=1in]{geometry}
\usepackage{microtype}
\usepackage{setspace}
\usepackage{amsmath}
\usepackage{parskip}
\usepackage{amssymb}

\parskip=4ex
\date{}
\author{}

\title{4.3 Linearly Independent Sets; Bases}

\begin{document}
  \maketitle
  An indexed set of vectors $ \{ v_1,...,v_p \} $ in $ V $ is said to be linearly independent if the linear equation
  \[
    c_1v_1+c_2v_2+...+c_pv_p=0
  \]
  has only the trivial solution.

  THe set $ \{ v_1,...,v_p \} $ is said to be linearly dependent if $ (1) $ has a nontrivial solution, that is, if there are some weights $ c_1,...,c_p $, not all zero, such that $ (1) $ holds. In such a case, $ (1) $ is called a linear dependence relation among $ v_1,...,v_p $.

  \textbf{Theorem 4.4}\\
  An indexed set $ \{ v_1,...,v_p \} $ of two or more vectors, with $ v_1 \neq 0 $ is linearly dependent if and only if some $ v_j $ (with $ j>1 $ ) is a linear combination of the preceding vectors $ v_1,...,v_{j-1} $.

  A main difference between linear dependence in $ \mathbb{R}^{n} $ and in a general vector space is that when the vectors are not $ n-tuples $, the homogeneous equation $ (1) $ usually cannot be written as a system of $ n $ linear equations. Meaning the vectors cannot be made into the columns of a matrix $ A $ in order to study the equation $ Ax=0 $. The definition of linear independence and Theorem 4.4 must be used.

  \textbf{Ex 1}\\
  Let $ p_1(t)=1, P_2(t)=t ~\&~ p_3(t)=4-t$. Then $ \{ p_1,p_2,p_3 \} $ is linearly dependent in $ \mathbb{P} $ because $ p_3=4p_1-p_2 $.

  \textbf{Def}\\
  Let $ H $ be a subspace of a vector space $ V $. A set of vectors $ \mathcal{B} $ in $ V $ is a basis for $ H $ if

  (i) $ \mathcal{B} $ is linearly independent set, and

  (ii) the subspace spanned by $ \mathcal{B} $ coincides with $ H $; that is,
  \[
    H=\text{Span }\mathcal{B}
  \]
  
  The definition of a basis applies to the case when $ H=V $, because any vector space is a subsapce of itself. Thus a basis of $ V $ is a linearly independent set that spans $ V $. 

  \textbf{Ex 3}\\
  Let $ A $ be an invertible $ n \times n $ matrix. Then the columns of $ A $ form a basis for $ \mathbb{R}^{n} $ because they are linearly independent and span $ \mathbb{R}^{n} $, by the Invertible Matrix Theorem.

  \textbf{Ex 5}\\
  Let $ v_1 = \begin{bmatrix}
    3\\
    0\\
    -6
  \end{bmatrix}, v_2 = \begin{bmatrix}
    -4\\
    1\\
    7
  \end{bmatrix}, ~\&~ v_3 = \begin{bmatrix}
    -2\\
    1\\
    5
  \end{bmatrix} $. Determine if $ \{ v_1,v_2,v_3 \} $ is a basis for $ \mathbb{R}^{3} $.
  \[
    A = \begin{bmatrix}
      3 &-4 &-2\\
      0 &1 &1\\
      -6 &7 &5
    \end{bmatrix} \to 
    \begin{bmatrix}
      1 &0 &0\\
      0 &1 &0\\
      0 &0 &1
    \end{bmatrix}
  \]

  The matrix $ A=\begin{bmatrix}
    v_1 &v_2 &v_3
  \end{bmatrix} $ has three pivot positions thus $ A $ is invertible. So the columns of $ A $ form a basis for $ \mathbb{R}^{3} $.

  \textbf{The Spanning Set Theorem}\\
  A basis is an "efficient" spanning set that contains no unnecessary vectors. A basis can be constructed from a spanning set by discarding unneeded vectors.

  \textbf{Ex 7}\\
  Let
  \[
    \begin{gathered}
    v_1 = \begin{bmatrix}
      0\\
      2\\
      -1
    \end{bmatrix}, v_2 =
    \begin{bmatrix}
      2\\
      2\\
      0
    \end{bmatrix}, v_3=
    \begin{bmatrix}
      6\\
      16\\
      -5
    \end{bmatrix}, ~\&~ 
    H = \text{Span }\{ v_1,v_2,v_3 \}
    \end{gathered}
  \]

  Note that $ v_3=5v_1+3v_2 $, and show that Span $ \{ v_1,v_2,v_3 \} = $ Span $ \{ v_1,v_2 \} $. Then find a basis for the subspace $ H $.

  Every vector in Span $ \{ v_1,v_2 \} $ belongs to $ H $ because
  \[
    c_1v_1+c_2v_2=c_1v_1+c_2v_2+0v_3
  \]
  
  Now let $ x $ be any vector in $ H $, $ x=c_1v_1+c_2v_2+c_3v_3 $. Since $ v_3=5v_1+3v_2 $, we may substitute
  \[
    \begin{gathered}
    x=c_1v_1+c_2v_2+c_3(5v_1+3v_2)\\
    (c_1+5c_3)v_1 + (c_2+3c_3)v_2
    \end{gathered}
  \]

  This $ x $ is in Span $ \{ v_1,v_2 \} $, so every vector in $ H $ already belongs to Span $ \{ v_1,v_2 \} $. Thus $ H $ and Span $ \{ v_1,v_2 \} $ are actually the same set of vectors.

  \textbf{The Spanning Set Theorem}\\
  Let $ S=\{ v_1,...,v_p \} $ be a set in a vector space $ V $, and let $ H = $ Span $ \{ v_1,...,v_p \} $

  a. If one of the vectors in $ S $ such as $ v_k $ is a linear combination of the remaining vectors in $ S $, then the set formed from $ S $ by removing $ v_k $ still spans $ H $.

  b. If $ H \neq \{ 0 \} $, some subset of $ S $ is a basis for $ H $. 

  \textbf{Ex 8}\\
  Find a basis for Col $ B $, where
  \[
    B = \begin{bmatrix}
      b_1 &b_2 &... &b_5
    \end{bmatrix} = 
    \begin{bmatrix}
      1 &4 &0 &2 &0\\
      0 &0 &1 &-1 &0\\
      0 &0 &0 &0 &1\\
      0 &0 &0 &0 &0
    \end{bmatrix}
  \]
  
  Each nonpivot column of $ B $ is a linear combination of the pivot columns. In fact, $ v_2=4b_1 ~\&~ b_4=2b_1-b_3 $. By the Spanning Set Theorem, we may discard $ v_2 ~\&~ b_4 $. By the Spanning Set Theorem, we may discard $ b_2 ~\&~ b_4 ~\&~ \{ b_1,b_3,b_5 \} $ will still span Col $ B $. Let
  \[
    S = \text{\Huge{\{}}
      \begin{bmatrix}
        1\\
        0\\
        0\\
        0
      \end{bmatrix},
      \begin{bmatrix}
        0\\
        1\\
        0\\
        0
      \end{bmatrix},
      \begin{bmatrix}
        0\\
        0\\
        1\\
        0
      \end{bmatrix}
    \text{\Huge{\}}}
  \]

  Since $ b_1 \neq 0 $ and no vector in $ S $ is a linear combination of the vectors that precide it, $ S $ is linearly independent. Thus $ S $ is a basis for Col $ B $.   
  
  \textbf{Theorem 4.6}\\
  The pivot columns of a matrix $ A $ form a basis for Col $ A $.


  
  
\end{document}
