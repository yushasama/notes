\documentclass{article}
\usepackage[margin=1in]{geometry}
\usepackage{microtype}
\usepackage{setspace}
\usepackage{amsmath}
\usepackage{parskip}
\usepackage{amssymb}
\usepackage{graphicx}

\graphicspath{{../public/}}

\parskip=4ex
\date{}
\author{}

\title{5.9 Applications to Markov Chains}

\begin{document}
  \maketitle
  \textbf{Definition}\\
  A vector with nonnegative entries that add up to 1 is called a probability vector. A stochastic matrix is a square matrix whose columns are probability vectors.

  A Markov chain is a sequence of probability vectors $ x_0,x_1,x_2,..., $ together with a stochastic matrix $ P $, such that
  \[
    \begin{gathered}
    x_1=Px_0, \qquad x_2=Px_1, \qquad x_3=Px_2, \qquad ...
    \end{gathered}
  \]

  Thus the Markov chain is described by the first-order difference equation
  \[
    x_{k+1}=Px_k \qquad \text{for }k=0,1,2...
  \]

  When a Markov chain of vectors in $ \mathbb{R}^{n} $ describes a system or a sequence of experiments, the entries in $ x_k $ list, respectively, the probabilities that the outcome of the experiment is one of possible $ n $ outcomes. For this reason, $ x_k $ is often called a state vector.

  \textbf{Ex 1}\\
  Given a migration matrix $ M $ 
  \[
    \begin{gathered}
    M=\begin{bmatrix}
      .95 &.03\\
      .05 &.97
    \end{bmatrix}
    \end{gathered}
  \]

  Wheere the columns represent the percentage of city \& surbaban residents moving to city and suburbs (depicted by the first and second row respectively).

  Suppouse that in $ 2020 $, the population is $ 600,000 $ in the city and $ 400,000 $ in the suburbs. What is the distribution of the population in $ 2021 $? In $ 2022 $?

  \[
    \begin{gathered}
    \begin{bmatrix}
      .95 &.03\\
      .05 &.97
    \end{bmatrix}
    \begin{bmatrix}
      600,000\\
      400,000
    \end{bmatrix}=
    \begin{bmatrix}
      582,000\\
      418,000
    \end{bmatrix}
    \end{gathered}
  \]

  Then by dividing the result by the total population of $ 1,000,000 $, we get
  $x_1= \begin{bmatrix}
    .582\\
    .418
  \end{bmatrix} $ 
  
  So the vector $ x_1 $ gives the population distribution in $ 2021 $. That is $ 58.2\% ~\&~ 41.8\% $ of the population lived in the city and suburbs, respectively. Similarly, the population distribution in $ 2022 $ is described by $ x_2 $, where
  \[
    \begin{gathered}
    x_2=Mx_1=\begin{bmatrix}
      .95 &.03\\
      .05 &.97
    \end{bmatrix}
    \begin{bmatrix}
      .582\\
      .418
    \end{bmatrix}=
    \begin{bmatrix}
      .565\\
      .435
    \end{bmatrix}
    \end{gathered}
  \]

  \textbf{Ex 1}\\
  Given the matrix $ A=\begin{bmatrix}
    -1.5 &.5\\
    1 &-1
  \end{bmatrix}, x(t)=\begin{bmatrix}
    x_1(t)\\
    x_2(t)
  \end{bmatrix}, ~\&~ x(0)=\begin{bmatrix}
    5\\
    4
  \end{bmatrix} $. 

  The eigenvalues of $ A $ are $ \lambda_1=-.5 ~\&~ \lambda_2=-2 $ with corresponding eigenvectors
  \[
    v_1=\begin{bmatrix}
      1\\
      2
    \end{bmatrix} \qquad
    v_2=\begin{bmatrix}
      -1\\
      1
    \end{bmatrix}
  \].


  
  
\end{document}
