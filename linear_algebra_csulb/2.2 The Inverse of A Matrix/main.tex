\documentclass{article}
\usepackage[margin=1in]{geometry}
\usepackage{microtype}
\usepackage{setspace}
\usepackage{amsmath}
\usepackage{parskip}
\usepackage{amssymb}
\usepackage{graphicx}

\graphicspath{{../public/}}

\parskip=4ex
\date{}
\author{}

\title{2.2 The Inverse of A Matrix}

\begin{document}
  \maketitle
  \textbf{Definition}\\
  An $ n \times n $ matrix $ A $ is said to be invertible if there is an $ n \times n $ matrix $ C $ such that $ AC = I ~\&~ CA=I$. Where $ I = I_{n} $ the $ n \times n $ identity matrix. 
  \[
    \begin{gathered}
    I=I_{2} = \begin{bmatrix}
      1 &0\\
      0 &1
    \end{bmatrix} = [Ie_{1},Ie_{2}  ] \qquad
    I = I_{3} = \begin{bmatrix}
      1 &0 &0\\
      0 &1 &0\\
      0 &0 &1
    \end{bmatrix} = [Ie_{1},Ie_{2},Ie_{3}   ]\\
    ~\\
    I=I_{n} = [Ie_{1}, Ie_{2},...Ie_{n}]  
    \end{gathered}
  \]
  
  \textbf{Algebraic Representation}\\
  $ (CA)x = Ix = x $, with this we can say that $ I $ is essentially "$ 1 $". 
  \[
    \begin{gathered}
    AA^{-1} = A^{-1}A = I  
    \end{gathered}
  \]

  \textbf{Ex 1}\\
  If $ A = \begin{bmatrix}
      2 &5\\
      -3 &-7
  \end{bmatrix} $ and $ C = \begin{bmatrix}
      -7 &-5\\
      3 &2
  \end{bmatrix} $, then
  \[
      AC = \begin{bmatrix}
          2 &5\\
          -3 &-7
      \end{bmatrix}
      \begin{bmatrix}
          -7 &-5\\
          3 &2
      \end{bmatrix} = 
      \begin{bmatrix}
          1 &0\\
          0 &1
      \end{bmatrix}
  \]
  and
  \[
      AC=\begin{bmatrix}
          -7 &-5\\
          3 &2
      \end{bmatrix}
      \begin{bmatrix}
          2 &5\\
          -3 &-7
      \end{bmatrix} = 
      \begin{bmatrix}
          1 &0\\
          0 &1
      \end{bmatrix}
  \]

  Thus $ C=A^{-1}  $, then in the context of the definition $ AC =I ~\&~ CA=I,C$ is the inverted matrix $ A^{-1}$.   
  
  \textbf{Theorem 4}\\
  $ A = \begin{bmatrix}
      a &b\\
      c &d
  \end{bmatrix} $. If the determinant $ |A|, ad-bc \neq 0$, then $ A $ is invertible and
  \[
      A^{-1}=\frac{1}{ad-bc} \begin{bmatrix}
          d &-b\\
          -c &a
      \end{bmatrix} 
  \]

  If $ ad-bc =0$, then $ A $ is not invertible. 

  \textbf{Ex 2}\\
  Find the inverse of $ A=\begin{bmatrix}
      3 &4\\
      5 &6
  \end{bmatrix} $
  \[
      \begin{gathered}
        |A|=3(6)-4(5)=-2\\
        ~\\
      A^{-1} = -\frac{1}{2}\begin{bmatrix}
          6 &-4\\
          -5 &3
      \end{bmatrix} =
      \begin{bmatrix}
          -3 &2\\
          \frac{5}{2}  &-\frac{3}{2} 
      \end{bmatrix}
      \end{gathered}
  \]
  
  \textbf{Theorem 5}\\
  If $ A $ is an invertible $ n \times n $ matrix, then for each $ b $ in $ \mathbb{R}^{n}$, the equation $ Ax=b $ has the unique solution $ x=A^{-1}b$.
  \[
    \begin{gathered}
      Ax=b \\ A^{-1}Ax = A^{-1}b \\ x=A^{-1}b
    \end{gathered}
  \]
  based on the fact that $ A^{-1}Ax =Ix = x$.

  \textbf{Ex 4}\\
  Use the inverse of the matrix $ A $ to solve the system
  \[
      \begin{gathered}
      3x_{1}+4x_{2}=3\\
      5x_{1}+6x_{2}=7    
      \end{gathered}
  \]
  This system is in the format $ Ax=b $, so
  \[
      \begin{gathered}
      x=A^{-1}b = 
      \begin{bmatrix}
          -3 &2\\
          \frac{5}{2} &-\frac{3}{2}  
      \end{bmatrix}
      \begin{bmatrix}
          3\\
          7
      \end{bmatrix} =
      \begin{bmatrix}
          5\\
          -3
      \end{bmatrix}
      \end{gathered}
  \]
  
  \textbf{Theorem 6}\\
  A) If $ A $ is an invertible matrix, then $ A^{-1}$ is invertible and
  \[
    (A^{-1})^{-1}=A 
  \]
  B) If $ A ~\&~ B $ are $ n \times n $ invertible matrices, then so is $ AB $, and the inverse of $ AB $ is the product of $ A ~\&~ B $ in the reverse order.
  \[
    (AB)^{-1}=B^{-1}A^{-1}   
  \]
  C) If $ A $  is an invertible matrix, then so is $ A^{T}  $, and the inverse of $ A^{T}  $ is the transpose of $ A^{-1}  $.
  \[
    (A^{T} )^{-1}=(A^{-1})^{T} 
  \]

  \textbf{Theorem 7}\\
  An $ n \times n $ matrix $ A $ is invertible if and only if $ A $ is row equivalent to $ I_{n}  $, and in this case, any sequence of elementary row operations that reduces $ A \to I_{n} $, also transforms $ I_{n}\to A^{-1} $.
  
  Recall that the inverse of $ A $ is $ A^{-1}  $  such that $ AA^{-1}=A^{-1}A=I$.

  \textbf{An Algorithm for Finding $ A^{-1} $ }\\
  By placing $ A ~\&~ I $ side by side to form an augmented matrix $ \begin{bmatrix}
    A &I
  \end{bmatrix} $, then row operations on this matrix will be applied to both $ A ~\&~ I $. Then according to Theorem 7, this means that there are row operations that transforms $ A \to I_{n} $ \&$ I_{n} \to A^{-1} $ or else $ A $ is not an invertible matrix.     
  
  \textbf{Ex 7}\\
  Find the inverse of the matrix $ A = \begin{bmatrix}
    0 &1 &2\\
    1 &0 &3\\
    4 &-3 &8
  \end{bmatrix} $ 
  \[
    \begin{gathered}
    A I = \begin{bmatrix}
      0 &1 &2 &1 &0 &0\\
      1 &0 &3 &0 &1 &0\\
      4 &-3 &8 &0 &0 &1
    \end{bmatrix} \to
    \begin{bmatrix}
      1 &0 &0 &-\frac{9}{2} &7 &-\frac{3}{2}\\
      0 &1 &0 &-2 &4 &-1\\
      0 &0 &1 &\frac{3}{2} &-2 &\frac{1}{2}  
    \end{bmatrix}
    \end{gathered}
  \]
  Since $ A\to I_{n} $, then $ A $ is invertible, giving us
  \[
    A^{-1} = \begin{bmatrix}
      \frac{-9}{2} &7 &-\frac{3}{2} \\
      -2 &4 &-1\\
      \frac{3}{2} &-2 &\frac{1}{2} 
    \end{bmatrix} 
  \]
\end{document}
