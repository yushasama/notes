\documentclass{article}
\usepackage[margin=1in]{geometry}
\usepackage{microtype}
\usepackage{setspace}
\usepackage{amsmath}
\usepackage{parskip}
\usepackage{amssymb}
\usepackage{graphicx}

\graphicspath{{../public/}}

\parskip=4ex
\date{}
\author{}

\title{1.5 Linear Independence}

\begin{document}
  \maketitle
  \textbf{Homogeneous Linear Systems}\\
  Given vectors $ \{v_{1},v_{2},...v_{p}\} \in \mathbb{R}^{n} $. We will solve the system of equations $ x_{1}v_{2}+x_{2}v_{2}+...+x_{p}v_{p}=0$. $ v_{1}v_{1}, ..., x_{p}v_{p}$ are vectors in $\mathbb{R}^{n}$ that we will be solving for and eventually give us values for $ x_{1},x_{2},...,x_{p}$.

  \textbf{Two Cases}
  \[
    \begin{gathered}
      1)~\text{If } x_{1}=0, x_{2}=0,...,x_{p}=0~\text{then we say that the set} \{v_{1},v_{2},...,v_{p}\} \text{ is linearly independent.} \\
      ~\\
      2)~\text{If } x_{1},...,x_{p} \text{ are not all zeros, then the set} \{v_{1},v_{2},...,v_{p}\} \text{ is linearly dependent.} 
    \end{gathered}
  \]

  \textbf{Ex 1 p60}
  \[
    \begin{gathered}
    v_{1}= \begin{bmatrix}
      1\\
      2\\
      3
    \end{bmatrix}
    v_{2} = \begin{bmatrix}
      4\\
      5\\
      6
    \end{bmatrix}
    v_{3}=\begin{bmatrix}
      2\\
      1\\
      0
    \end{bmatrix} \\
    \end{gathered}
  \]
  
  1A. Determine if the set $ \{v_{1},v_{2},v_{3}\} $ is linearly independent.
  2B. If possible find a linear dependence relation among $ v_{1},v_{2},v_{3}$.
\[
  \begin{gathered}
  \text{A) }
  x_{1}v_{1}+x_{2}v_{2}+x_{3}v_{3}=0, \qquad v_{1},...v_{p}  \text{are vectors in } \mathbb{R}^{3}.\\
  ~\\
  \begin{bmatrix}
    1 &4 &2 &0\\
    2 &5 &1 &0\\
    3 &6 &0 &0
  \end{bmatrix} \to 
  \begin{bmatrix}
    1 &4 &2 &0\\
    0 &-3 &3 &0\\
    0 &0 &0 &0
  \end{bmatrix}\to 
  \begin{bmatrix}
    1 &4 &2 &0\\
    0 &1 &1 &0\\
    0 &0 &0 &0
  \end{bmatrix}\\
  ~\\
  x_{1},~x_{2}=\text{pivot},~ x_{3} = \text{free}\\
  ~\\
  x_{3}=1,~
  x_{2}=-1,~
  x_{1}=-4x_{2}-2x_{3}\to 2
  ~\\
  (2,-1,1) \text{ is a solution. So the set } \{v_{1},v_{2},v_{3}\} \text{ is linearly dependent.}\\
  ~\\
  \text{B) Linear Dependence Relation}\\
  ~\\
  2v_{1}-v_{2}+v_{3}=0   
  \end{gathered}\\
  \]

  \textbf{Ex 2 p61}\\
  Determine if the columns of a matrix, $ A $ are linearly independent.
  \[
    \begin{gathered}
    A=\begin{bmatrix}
      0 &1 &4\\
      1 &2 &-2\\
      5 &8 &0
    \end{bmatrix}\\
    ~\\
    x_{1}c_{1}+x_{2}c_{2}+x_{3}c_{3}=0, \qquad \text{is what we are solving for where } c_{n} \text{ represents a column.}\\
    ~\\
    x_{1}\begin{bmatrix}
      0\\
      1\\
      5
    \end{bmatrix} +
    x_{2}\begin{bmatrix}
      1 \\
      2\\
      8
    \end{bmatrix} +x_{3} \begin{bmatrix}
      4\\
      -1\\
      0
    \end{bmatrix}  = \begin{bmatrix}
      0\\
      0\\
      0
    \end{bmatrix}\\
    ~\\
    \begin{bmatrix}
      0 &1 &4\\
      1 &2 &-1\\
      5 &8 &0
    \end{bmatrix}
    \begin{bmatrix}
      x_{1}\\
      x_{2}\\
      x_{3}  
    \end{bmatrix} =
    \begin{bmatrix}
      0\\
      0\\
      0
    \end{bmatrix}\\
    ~\\
    Ax=0 \\
    ~\\
    \begin{bmatrix}
      0 &1 &4 &0\\
      1 &2 &-1 &0\\
      5 &8 &0 &0
    \end{bmatrix}\to
    \begin{bmatrix}
      1 &0 &-1 &0\\
      0 &1 &4 &0\\
      0 &0 &13 &0
    \end{bmatrix}\\
    ~\\
  x_{3}=0,~x_{2}=0,~x_{1}=0,~\text{linearly independent} 
    \end{gathered}
  \]
  Due to no free variables existing. Meaning that there is only a trivial solution for the equation $Ax=0$ thus the columns of $A$ are linearly independent.
\end{document}
