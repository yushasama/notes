\documentclass{article}
\usepackage[margin=1in]{geometry}
\usepackage{microtype}
\usepackage{setspace}
\usepackage{amsmath}
\usepackage{parskip}
\usepackage{amssymb}

\parskip=4ex
\date{}
\author{}

\title{2.3 Characterizations of Invertible Matrices}

\begin{document}
    \maketitle
    \textbf{The Invertible Matrix Theorem}\\
    Let $ A $ be a square $ n \times n $ matrix. Then the following statements are equivalent. That is, for a given $ A $, the statements are either all true or all false.

    a) $ A $ is an invertible matrix.\\
    b) $ A $ is row equivalent to the $ n \times n $ identity matrix.\\
    c) $ A $ has $ n $ pivot positions.\\
    d) The equation $ Ax=0 $ has only the trivial solution.\\
    e) The columns of $ A $ form a linearly independent set.\\
    f) The linear transformations $ x \mapsto Ax $ is one-to-one..\\
    g) The equation $ Ax=b $ has at least one solution for each $ b $ in $ \mathbb{R}^{n}  $.\\
    h) The columns of $ A $ span $ \mathbb{R}^{n}  $.\\
    i) The linear transformation $ x \mapsto Ax $ maps $ \mathbb{R}^{n}  $ onto $ \mathbb{R}^{n}  $.\\
    j) There is an $ n \times n $ matrix $ C $ such that $ CA=I $.\\
    k) There is an $ n \times n $ matrix $ D $ such that $ AD=I $.\\
    l) $ A^{T}  $ is an invertible matrix.

    $ (a) \implies (j) \implies (d) \implies (c) \implies (b) \implies (a) $, this would mean if any of these five statements hold true, so do the rest of them. Then the remaining statements of the Invertible Matrix Theorem will also hold true.

    \textbf{Ex 1}\\
    Use the Invertible Matrix Theorem to decide if $ A $ is invertible
    \[
        \begin{gathered}
        A = \begin{bmatrix}
            1 &0 &-2\\
            3 &1 &-2\\
            -5 &-1 &9
        \end{bmatrix}\to 
        \begin{bmatrix}
            1 &0 &-2\\
            0 &1 &4\\
            0 &0 &3
        \end{bmatrix}
        \end{gathered}
    \]
    Since $ A $ has three pivot positions while being a $ 3 \times 3 $ matrix, then the matrix $ A $ is invertible, by statement $ (c) $ of the Invertible Matrix Theorem.

    \textbf{Invertible Linear Transformations}\\
    When a matrix $ A $ is invertible, the equation $ A^{-1}Ax =x $  can be viewed as a statement about linear transformations.

    A linear transformation $ T:\mathbb{R}^{n}\to \mathbb{R}^{n} $ is said to be invertible if there exists a function $ S:\mathbb{R}^{n} \to \mathbb{R}^{n} $ such that
    \[
        \begin{gathered}
          S(T(x))=x \qquad \forall x \in \mathbb{R}^{n}\\
          T(S(x))=x \qquad \forall x \in \mathbb{R}^{n}  
        \end{gathered}
    \]

    The next theorem shows that if such an $ S $ exists, it is unique and must be a linear transformation. We call $ S $ the inverse of $ T $ and write it as $ T^{-1}  $ .

    \textbf{Theorem 9}\\
    Let $ T:\mathbb{R}^{n} \to \mathbb{R}^{n} $ be a linear transformation and let $ A $ be the standard matrix for $ T. $ Then $ T $ is invertible if and only if $ A $ is an invertible matrix. In that case, the linear transformation given by $ S(x) =A^{-1x} =A^{-1} x$ is the unique function the equations
    \[
        \begin{gathered}
        S(T(x))=x \qquad \forall x \in \mathbb{R}^{n}\\
        T(S(x))=x \qquad \forall x \in \mathbb{R}^{n}  
        \end{gathered}
    \]

    \textbf{Ex 2}\\
    What can you say about a one-to-one linear transformation $ T $ from $ \mathbb{R}^{N} $ into $ \mathbb{R}^{n}  $?

    The columns of the standard matrix of $ A $ of $ T $ are linearly independent. So $ A $ is invertible, then by the Invertible Matrix Theorem and $ T $ maps $ \mathbb{R}^{n}  $ onto $ \mathbb{R}^{n}  $. Also by Theorem 9, then $ T $ is invertible.  
    
\end{document}
